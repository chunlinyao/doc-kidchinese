%# -*- coding: utf-8 -*-
%!TEX encoding = UTF-8 Unicode
%!TEX TS-program = xelatex
% vim:ts=4:sw=4
%
% 以上设定默认使用 XeLaTex 编译,并指定 Unicode 编码,供 TeXShop 自动识别
\documentclass[avery5371,grid]{flashcards}


\cardfrontstyle[\large\slshape]{headings}
\cardbackstyle{empty}
%\cardfrontfoot{中文}
\cardfrontfoot{}%四五快读5

\newcommand{\doctitle}{四五快读 5}
\newcommand{\docauthor}{Yunhui Fu}
\newcommand{\dockeywords}{{中文}{四五快读}}
\newcommand{\docsubject}{}

% debug
\newcommand{\SecDanzi}[1]{#1} % 单字
\newcommand{\SecCi}[1]{#1} % 词
\newcommand{\SecJuzi}[1]{#1} % 句子

%\renewcommand{\SecDanzi}[1]{} % 单字
%\renewcommand{\SecCi}[1]{} % 词
%\newcommand{\comments}[1]{} % 句子


%# -*- coding: utf-8 -*-
%!TEX encoding = UTF-8 Unicode
%!TEX TS-program = xelatex
% vim:ts=4:sw=4
%
% 以上设定默认使用 XeLaTex 编译,并指定 Unicode 编码,供 TeXShop 自动识别


\newcommand\mymainfont{Noto Serif}%{Times New Roman} %{DejaVu Serif}
\newcommand\mymonofont{DejaVu Sans Mono}%{FreeMono} %{WenQuanYi Micro Hei Mono} %{Monaco}
\newcommand\myboldfont{DejaVu Sans Mono} %{WenQuanYi Micro Hei Mono}%{AR PL UKai CN}%{YaHei Consolas Hybrid}%{黑体}%{標楷體}
\newcommand\mysansfont{DejaVu Sans}%{FreeSans}
\newcommand\myitalicfont{DejaVu Serif}%{Times New Roman} %{Garamond}

\newcommand\mycjkboldfont{Microsoft YaHei} %{WenQuanYi Micro Hei Mono}%{Adobe Heiti Std}%{AR PL UKai CN}%{YaHei Consolas Hybrid}%{黑体}%{標楷體}
\newcommand\mycjkitalicfont{全字庫正楷體} %{Adobe Kaiti Std}
\newcommand\mycjkmainfont{WenyueType GutiFangsong (Noncommercial Use)}%{全字庫正楷體} %{Adobe Song Std} %{AR PL UMing CN}%{仿宋}%{宋体}%{新宋体}%{文鼎PL新宋}%
\newcommand\mycjksansfont{MingLiU} %{Adobe Ming Std}
\newcommand\mycjkmonofont{DejaVu Sans YuanTi Mono}%{WenQuanYi Micro Hei Mono}%{AR PL UMing CN}%{WenQuanYi Micro Hei Mono}


%\usepackage[nofonts]{ctex} %adobefonts
\usepackage[adobefonts]{ctex} %adobefonts
%\usepackage[fallback]{xeCJK}


\newCJKfontfamily{\mykaiti}{全字庫正楷體}%{AR PL UKai TW}%{全字庫正楷體}
\newCJKfontfamily{\myfangsong}{WenyueType GutiFangsong (Noncommercial Use)}



\usepackage{ifthen}
\usepackage{ifpdf}
\usepackage{ifxetex}
\usepackage{ifluatex}


\usepackage{color}
\usepackage[rgb]{xcolor}



\xeCJKsetup{AutoFallBack = true} % you have to use this, since fallback won't work as the xeCJK option after the ctex

\PassOptionsToPackage{
    BoldFont,  % 允許粗體
    SlantFont, % 允許斜體
    CJKnumber,
    CJKtextspaces,
    }{xeCJK}
\defaultfontfeatures{Mapping=tex-text} % 如果沒有它,會有一些 tex 特殊字符無法正常使用,比如連字符。

\xeCJKsetup {
    CheckSingle = true,
    AutoFakeBold = false,
AutoFakeSlant = false,
    CJKecglue = {},
    PunctStyle = kaiming,
KaiMingPunct+ = {:;},
}

\PassOptionsToPackage{CJKchecksingle}{xeCJK}
%\defaultCJKfontfeatures{Scale=0.5}
%\LoadClass[c5size,openany,nofonts]{ctexbook}
\XeTeXlinebreaklocale "zh"                      % 重要,使得中文可以正確斷行!
\XeTeXlinebreakskip = 0pt plus 1pt minus 0.1pt  % 给予TeX断行一定自由度
\linespread{1.3}                                % 1.3倍行距

\setCJKmainfont[BoldFont=\mycjkboldfont, ItalicFont=\mycjkitalicfont]{\mycjkmainfont}
\setCJKsansfont{\mycjksansfont}%{Adobe Ming Std} %{AR PL UMing CN} %{Microsoft YaHei}
\setCJKmonofont{\mycjkmonofont}



\ifxetex % xelatex
\else
    %The cmap package is intended to make the PDF files generated by pdflatex "searchable and copyable" in acrobat reader and other compliant PDF viewers.
    \usepackage{cmap}%
\fi
% ============================================
% Check for PDFLaTeX/LaTeX
% ============================================
\newcommand{\outengine}{xetex}
\newif\ifpdf
\ifx\pdfoutput\undefined
  \pdffalse % we are not running PDFLaTeX
  \ifxetex
    \renewcommand{\outengine}{xetex}
  \else
    \renewcommand{\outengine}{dvipdfmx}
  \fi
\else
  \pdfoutput=1 % we are running PDFLaTeX
  \pdftrue
  \usepackage{thumbpdf}
  \renewcommand{\outengine}{pdftex}
  \pdfcompresslevel=9
\fi
\usepackage[\outengine,
    bookmarksnumbered, %dvipdfmx
    %% unicode, %% 不能有unicode选项,否则bookmark会是乱码
    colorlinks=true,
    citecolor=red,
    urlcolor=blue,        % \href{...}{...} external (URL)
    filecolor=red,      % \href{...} local file
    linkcolor=black, % \ref{...} and \pageref{...}
    breaklinks,
    pdftitle={\doctitle},
    pdfauthor={\docauthor},
    pdfsubject={\docsubject},
    pdfkeywords={\dockeywords},
    pdfproducer={Latex with hyperref},
    pdfcreator={pdflatex},
    %%pdfadjustspacing=1,
    pdfborder=1,
    pdfpagemode=UseNone,
    pagebackref,
    bookmarksopen=true]{hyperref}

% --------------------------------------------
% Load graphicx package with pdf if needed 
% --------------------------------------------
\ifxetex    % xelatex
    \usepackage{graphicx}
\else
    \ifpdf
        \usepackage[pdftex]{graphicx}
        \pdfcompresslevel=9
    \else
        \usepackage{graphicx} % \usepackage[dvipdfm]{graphicx}
    \fi
\fi

%\newCJKfontfamily{\yanti}{\mycjkmainfont}
%\newenvironment{shicibody}{%
%\begin{verse}\centering\yanti\large\hspace{13pt}}{\end{verse}}


\usepackage{xstring}

%The baseline-skip should be set to roughly 1.2x the font size.
%\fontsize{size}{baselineskip}
%\fontsize{50}{60}

% 计算所能容纳的字数
\newcommand\shizishow[1]{
\StrLen{#1}[\mystringlen]
%\def\mythresh{1}
    \ifthenelse{\mystringlen < 2}{
        {\fontsize{130}{156} #1}
    }{ \ifthenelse{\mystringlen < 3}{
        {\fontsize{110}{122} #1}%{\fontsize{100}{120} #1}
    }{ \ifthenelse{\mystringlen < 4}{
        {\fontsize{78}{94} #1}%{\fontsize{90}{108} #1}
    }{ \ifthenelse{\mystringlen < 5}{
        {\fontsize{55}{66} #1}
    }{ \ifthenelse{\mystringlen < 7}{
        {\fontsize{70}{84} #1}
    }{ \ifthenelse{\mystringlen < 9}{
        {\fontsize{55}{66} #1}
    }{ \ifthenelse{\mystringlen < 16}{
        {\zihao{0} #1} %5x4=20
    }{ \ifthenelse{\mystringlen < 49}{
        {\zihao{1} #1} % 8x6=48
    }{ \ifthenelse{\mystringlen < 55}{
        {\Huge #1} % 9x6=54
    }{ \ifthenelse{\mystringlen < 61}{
        {\zihao{2} #1} % 10x6=60
    }{ \ifthenelse{\mystringlen < 78}{
        {\huge #1} % 11x7=77
    }{ \ifthenelse{\mystringlen < 105}{
        {\LARGE #1} % 13x8=104
    }{ \ifthenelse{\mystringlen < 113}{
        {\zihao{3} #1} % 14x8=112
    }{ \ifthenelse{\mystringlen < 129}{
        {\Large #1} % 16x8=128 %{\zihao{4} #1} % 16x8=128
    }{ \ifthenelse{\mystringlen < 153}{
        {\large #1} % 19x8=152, or 19x9=171
    }{ \ifthenelse{\mystringlen < 198}{
        {\zihao{5} #1} % 22x9=198
    }{ \ifthenelse{\mystringlen < 207}{
        {\normalsize #1} % 23x9=207
    }{ \ifthenelse{\mystringlen < 234}{
        {\small #1} % 26x9=234
    }{ \ifthenelse{\mystringlen < 261}{
        {\footnotesize #1} % 29x9=261
    }{ \ifthenelse{\mystringlen < 279}{
        {\zihao{6} #1} % 31x9=279
    }{ \ifthenelse{\mystringlen < 297}{
        {\scriptsize #1} % 33x9=297
    }{ \ifthenelse{\mystringlen < 378}{
        {\zihao{7} #1} % 42x9=378
    }{
        {\tiny #1} % 46x9=414 %{\zihao{8} #1} % 46x9=414
    }}}}}}}}}}}}}}}}}}}}}}
}

% 计算包含拼音时所能容纳的字数
\newcommand\shizipy[1]{
\StrLen{#1}[\mystringlen]
\xpinyin*{
    \ifthenelse{\mystringlen < 6}{
        {\zihao{0} #1} %5x4=20
    }{ \ifthenelse{\mystringlen < 18}{
        {\zihao{1} #1} % 8x6=48
    }{ \ifthenelse{\mystringlen < 22}{
        {\Huge #1} % 9x6=54
    }{ \ifthenelse{\mystringlen < 31}{
        {\zihao{2} #1} % 10x6=60
    }{ \ifthenelse{\mystringlen < 34}{
        {\huge #1} % 11x7=77
    }{ \ifthenelse{\mystringlen < 53}{
        {\LARGE #1} % 13x8=104
    }{ \ifthenelse{\mystringlen < 57}{
        {\zihao{3} #1} % 14x8=112
    }{ \ifthenelse{\mystringlen < 65}{
        {\Large #1} % 16x8=128 %{\zihao{4} #1} % 16x8=128
    }{ \ifthenelse{\mystringlen < 77}{
        {\large #1} % 19x8=152, or 19x9=171
    }{ \ifthenelse{\mystringlen < 89}{
        {\zihao{5} #1} % 22x9=198
    }{ \ifthenelse{\mystringlen < 93}{
        {\normalsize #1} % 23x9=207
    }{ \ifthenelse{\mystringlen < 105}{
        {\small #1} % 26x9=234
    }{ \ifthenelse{\mystringlen < 117}{
        {\footnotesize #1} % 29x9=261
    }{ \ifthenelse{\mystringlen < 125}{
        {\zihao{6} #1} % 31x9=279
    }{ \ifthenelse{\mystringlen < 133}{
        {\scriptsize #1} % 33x9=297
    }{ \ifthenelse{\mystringlen < 169}{
        {\zihao{7} #1} % 42x9=378
    }{
        {\tiny #1} % 46x9=414 %{\zihao{8} #1} % 46x9=414
    }}}}}}}}}}}}}}}}
}
}


% 计算在comment中(三字经)能容纳的字数
\newcommand\sanzicomments[1]{
\StrLen{#1}[\mystringlen]
%\def\mythresh{1}
    \ifthenelse{\mystringlen < 7}{
        {\zihao{0} #1} %5x4=20
    }{ \ifthenelse{\mystringlen < 22}{
        {\zihao{1} #1} % 8x6=48
    }{ \ifthenelse{\mystringlen < 25}{
        {\Huge #1} % 9x6=54
    }{ \ifthenelse{\mystringlen < 38}{
        {\zihao{2} #1} % 10x6=60
    }{ \ifthenelse{\mystringlen < 42}{ %
        {\huge #1} % 11x7=77
    }{ \ifthenelse{\mystringlen < 63}{
        {\LARGE #1} % 13x8=104
    }{ \ifthenelse{\mystringlen < 64}{
        {\zihao{3} #1} % 14x8=112
    }{ \ifthenelse{\mystringlen < 87}{
        {\Large #1} % 16x8=128 %{\zihao{4} #1} % 16x8=128
    }{ \ifthenelse{\mystringlen < 125}{
        {\large #1} % 19x8=152, or 19x9=171
    }{ \ifthenelse{\mystringlen < 173}{
        {\zihao{5} #1} % 22x9=198
    }{ \ifthenelse{\mystringlen < 205}{
        {\normalsize #1} % 23x9=207
    }{ \ifthenelse{\mystringlen < 232}{
        {\small #1} % 26x9=234
    }{ \ifthenelse{\mystringlen < 259}{
        {\footnotesize #1} % 29x9=261
    }{ \ifthenelse{\mystringlen < 277}{
        {\zihao{6} #1} % 31x9=279
    }{ \ifthenelse{\mystringlen < 295}{
        {\scriptsize #1} % 33x9=297
    }{ \ifthenelse{\mystringlen < 376}{
        {\zihao{7} #1} % 42x9=378
    }{
        {\tiny #1} % 46x9=414 %{\zihao{8} #1} % 46x9=414
    }}}}}}}}}}}}}}}}
}



% 诗词
\newcommand\shici[5]{
\begin{flashcard}[#1 -- #2 (#3) #4]{%
{\mykaiti \fontsize{20}{24} {#5}} %\\
}
\vspace*{\stretch{1}}
\begin{center}

{\mykaiti \zihao{1} {#2}}\\ \vspace*{\stretch{.5}}
{\large(#3)}\\ \vspace*{\stretch{.25}}
{\LARGE #4}

\end{center}
\vspace*{\stretch{1}}
\end{flashcard}
}



% 识字

%\usepackage[overlap,CJK]{ruby}
\usepackage{xpinyin}

% 如果字数多,则略写
\newcommand\shizititle[2]{
  \StrLen{(#1) #2}[\mystringlen]%
  \ifthenelse{\mystringlen > 24}{
    \StrLeft{(#1) #2}{20}......\StrRight{(#1) #2}{4}
  }{
  (#1) #2
  }
}
\newcommand\shizi[3]{
\begin{flashcard}[\shizititle{#1}{#2}]{ %
{\mykaiti \shizipy{#2}}

{\myfangsong \normalsize #3} %
} %
\vspace*{\stretch{1}}
\begin{center}
%ēéěè
%\ruby{莉}{li}
%\xpinyin*{床前明月光,疑是地上霜。举头望明月,低头思故乡。}
%\begin{pinyinscope}
%床前明月光,疑是地上霜。举头望明月,低头思故乡。
%\end{pinyinscope}

{\mykaiti \shizishow{#2}}

\end{center}
\vspace*{\stretch{1}}
\end{flashcard}
}


\newcounter{fcsanzi}\setcounter{fcsanzi}{0}

% 三字经
\newcommand\sanzijing[3]{
\stepcounter{fcsanzi}
\begin{flashcard}[]{
\mykaiti \fontsize{26}{\baselineskip}\selectfont \xpinyin*{#1}
}
\vspace*{\stretch{1}}
%\begin{center}
%\normalsize
{
\myfangsong
\sanzicomments{
【解释】 #2

〖解读〗 #3
}
}
%\end{center}
\vspace*{\stretch{1}}
\end{flashcard}
}





\newcommand\docshowcopyright{
\begin{flashcard}[Copyright \& License]{Copyright \copyright \, 2014 Yunhui Fu\\
Some rights reserved.}
\vspace*{\stretch{1}}
These flashcards and the accompanying \LaTeX \, source code are licensed
under a Creative Commons Attribution--NonCommercial--ShareAlike 2.5 License.  
For more information, see creativecommons.org.  You can contact the author at:
\begin{center}
\begin{small}\tt yhfudev at gmail com\end{small}

\medskip
File last updated on \today, \\
at \currenttime
\end{center}
\vspace*{\stretch{1}}
\end{flashcard}

\begin{flashcard}[版权申明]{版权所有 \copyright \, 2014 Yunhui Fu\\
有些版权保留.}
\vspace*{\stretch{1}}
闪卡及其 \LaTeX \, 源代码在
署名-相同方式共享 2.5 下保护.
参见 creativecommons.org.  你可以联系作者:
\begin{center}
\begin{small}\tt yhfudev at gmail com\end{small}

\medskip
文件最近更新: \today, \\
\currenttime
\end{center}
\vspace*{\stretch{1}}
\end{flashcard}
}






\newcommand\docshowtitle[3]{
\begin{flashcard}[]{ %
{\mykaiti \Huge{#1}}

{\myfangsong \normalsize #2} %
} %
\vspace*{\stretch{1}}
\begin{center}

{\mykaiti \shizishow{#3}}

\end{center}
\vspace*{\stretch{1}}
\end{flashcard}
}








\usepackage{datetime}

\begin{document}

\docshowcopyright
\docshowtitle{\doctitle}{\docauthor}{%
使用双面打印,然后按线剪下。
}

\SecDanzi{
\shizi{5-41}{食}{}
\shizi{5-41}{无}{}
\shizi{5-41}{名}{}
\shizi{5-41}{加}{}
\shizi{5-41}{共}{}
\shizi{5-41}{事}{}
\shizi{5-41}{帮}{}
\shizi{5-41}{饿}{}
\shizi{5-41}{肚}{}
\shizi{5-41}{狮}{}
\shizi{5-41}{觉}{}
\shizi{5-41}{毛}{}
\shizi{5-41}{求}{}
\shizi{5-41}{等}{}
\shizi{5-41}{香}{}
\shizi{5-41}{肉}{}
}

% 宝宝读词语
\SecCi{
\shizi{5-41}{冷食}{}
\shizi{5-41}{甜食}{}
\shizi{5-41}{食指}{}
\shizi{5-41}{中指}{}
\shizi{5-41}{无风}{}
\shizi{5-41}{无边}{}
\shizi{5-41}{无能}{}
\shizi{5-41}{无心}{}
\shizi{5-41}{名字}{}
\shizi{5-41}{名人}{}
\shizi{5-41}{名山}{}
\shizi{5-41}{无名指}{}
\shizi{5-41}{加上}{}
\shizi{5-41}{加班}{}
\shizi{5-41}{一共}{}
\shizi{5-41}{共同}{}
\shizi{5-41}{共有}{}
\shizi{5-41}{做事}{}
\shizi{5-41}{大事}{}
\shizi{5-41}{小事}{}
\shizi{5-41}{好事}{}
\shizi{5-41}{事后}{}
\shizi{5-41}{有事}{}
\shizi{5-41}{公共}{}
\shizi{5-41}{帮我}{}
\shizi{5-41}{帮你}{}
\shizi{5-41}{帮他}{}
\shizi{5-41}{帮手}{}
\shizi{5-41}{帮工}{}
}

% 宝宝读短文
\SecJuzi{
\shizi{5-41}{一只手有五个手指:大拇指、食指、中指、无名指,还有小指。
左手和右手加起来,一共有十个手指。大家在一起很快乐,做什么事都是你帮我,我帮他,做得都很好。
有一天,大拇指生了气,它大声对食指、中指、无名指和小指说:“你们不要再来找我,做事不要找我。玩,也不要来找我。”
宝宝读短文
大家看大拇指生了这么大的气,也不敢找它。
到了吃饭时,它还在生气。
不好了,大拇指不和大家一起做事,就什么东西也拿不到手上。
饭也吃不到口里。吃不到东西,大家只好饿肚子了。}{}
}


\SecDanzi{
\shizi{5-42}{张}{}
\shizi{5-42}{网}{}
\shizi{5-42}{咬}{}
\shizi{5-42}{力}{}
\shizi{5-42}{啊}{}
\shizi{5-42}{牙}{}
\shizi{5-42}{嘴}{}
\shizi{5-42}{漂}{}
}

% 宝宝读词语
\SecCi{
\shizi{5-42}{肚子}{}
\shizi{5-42}{狮子狗}{}
\shizi{5-42}{毛毛}{}
\shizi{5-42}{要求}{}
\shizi{5-42}{求救}{}
\shizi{5-42}{高等}{}
\shizi{5-42}{饿了}{}
\shizi{5-42}{睡觉}{}
\shizi{5-42}{毛毛虫}{}
\shizi{5-42}{请求}{}
\shizi{5-42}{等一等}{}
\shizi{5-42}{中等}{}
\shizi{5-42}{肚子饿}{}
\shizi{5-42}{毛笔}{}
\shizi{5-42}{毛毛雨}{}
\shizi{5-42}{求人}{}
\shizi{5-42}{等等我}{}
\shizi{5-42}{好香}{}
\shizi{5-42}{狮子}{}
\shizi{5-42}{毛虫}{}
\shizi{5-42}{求求你}{}
\shizi{5-42}{求学}{}
\shizi{5-42}{等到}{}
\shizi{5-42}{香甜}{}
\shizi{5-42}{香花}{}
\shizi{5-42}{香水}{}
\shizi{5-42}{香气}{}
\shizi{5-42}{吃肉}{}
\shizi{5-42}{羊肉}{}
\shizi{5-42}{牛肉}{}
}

% 宝宝读短文
\SecJuzi{
\shizi{5-42}{在大草原上,一头狮子正在睡觉。一只老鼠从这里走过,还在狮子头上的毛里坐了一会儿。
“是谁!”狮子醒了,生气地一把捉住了老鼠:“原来是你,也好,正好做我的午饭。”
老鼠哭着求狮子:“放了我吧,我还有七个孩子在家等着我呢!你要是放了我,我会好好谢你,我是不会说假话的。”
狮子想了想,想到老鼠的七个孩子还在等妈妈,就把老鼠放掉了。
过了一天。狮子出来找吃的东西。
“好香呀!”狮子走过去,看到地上有一堆肉。正想去拿,一张大网一下子网住了它。动也不能动,狮子急得又喊又叫。
这时,正好鼠妈妈和她的孩子们从这儿走过。
“呀!狮子大哥,你怎么在这里面呀?只有我能救你了。来!孩子们,我们一起来咬开这张网。”
咬啊,咬啊,老鼠妈妈和小老鼠们把大网咬开了。狮子从里面跳了出来。
“谢谢你,老鼠太太。谢谢你们,老鼠小朋友。”}{}
}


\SecDanzi{
\shizi{5-43}{胡}{}
\shizi{5-43}{虎}{}
\shizi{5-43}{贴}{}
\shizi{5-43}{才}{}
\shizi{5-43}{数}{}
\shizi{5-43}{更}{}
\shizi{5-43}{朵}{}
\shizi{5-43}{纸}{}
}

% 宝宝读词语
\SecCi{
\shizi{5-43}{一张}{}
\shizi{5-43}{大网}{}
\shizi{5-43}{上网}{}
\shizi{5-43}{不咬}{}
\shizi{5-43}{用力}{}
\shizi{5-43}{来啊}{}
\shizi{5-43}{张口}{}
\shizi{5-43}{网子}{}
\shizi{5-43}{网球}{}
\shizi{5-43}{咬牙}{}
\shizi{5-43}{力气}{}
\shizi{5-43}{走啊}{}
\shizi{5-43}{张大}{}
\shizi{5-43}{渔网}{}
\shizi{5-43}{咬人}{}
\shizi{5-43}{咬东西}{}
\shizi{5-43}{大力水手}{}
\shizi{5-43}{吃啊}{}
\shizi{5-43}{长牙}{}
\shizi{5-43}{掉牙}{}
\shizi{5-43}{小嘴}{}
\shizi{5-43}{张嘴}{}
\shizi{5-43}{漂亮}{}
\shizi{5-43}{门牙}{}
\shizi{5-43}{月牙}{}
\shizi{5-43}{嘴里}{}
\shizi{5-43}{漂着}{}
\shizi{5-43}{虫牙}{}
\shizi{5-43}{尖牙}{}
\shizi{5-43}{嘴巴}{}
\shizi{5-43}{漂白}{}
}


% 宝宝读故事
\SecJuzi{
\shizi{5-43}{%大灰狼拔牙
大灰狼的牙疼,它就去找大猩猩。大猩猩问它:“你怎么啦?” 
大灰狼说:“我的牙疼得很,咬不动东西,天天都觉得肚子饿。很不好过。怎么办呢?”
大猩猩说:“我来给你拔掉吧!”
大灰狼说:“怎么拔呢?”
大猩猩想了想,说:“是呀!用什么东西拔呢?”
正在这时,小猴子来了,大猩猩就请小猴子帮着想。
小猴子想了想说:“有了。”就去拿了块石头,让大灰狼用力咬。
大灰狼不敢咬,说:“你这哪里是给我拔牙,你这是在害我呀!”
小猴子听了,回头就走。大猩猩也要走了。
大灰狼着了急,想快把痛的牙拔了去,好去找东西吃。
“好吧,好吧。我咬,我咬。是不是能拔掉呢?”
小猴子说:“你要用最大的力气咬,不用力是拔不掉的。”
大灰狼就用了最大的力气,大口一咬。
“啊呀!啊呀!不得了啦,好痛呀!”大灰狼一边喊,一边叫,一边跳。原来,大灰狼一嘴的牙都咬掉了。
灰狼再也吃不了肉了,再也没有谁怕它了。}{}
}




\SecDanzi{
\shizi{5-44}{圆}{}
\shizi{5-44}{圈}{}
\shizi{5-44}{亲}{}
\shizi{5-44}{脸}{}
\shizi{5-44}{眼}{}
\shizi{5-44}{睛}{}
\shizi{5-44}{接}{}
\shizi{5-44}{外}{}
}


% 宝宝读词语
\SecCi{
\shizi{5-44}{胡说八道}{}
\shizi{5-44}{老虎}{}
\shizi{5-44}{贴上}{}
\shizi{5-44}{才子}{}
\shizi{5-44}{数目}{}
\shizi{5-44}{更大}{}
\shizi{5-44}{胡说}{}
\shizi{5-44}{胡同}{}
\shizi{5-44}{贴着}{}
\shizi{5-44}{天才}{}
\shizi{5-44}{数字}{}
\shizi{5-44}{数来宝}{}
\shizi{5-44}{更好}{}
\shizi{5-44}{胡子}{}
\shizi{5-44}{胡来}{}
\shizi{5-44}{虎牙}{}
\shizi{5-44}{人才}{}
\shizi{5-44}{才能}{}
\shizi{5-44}{数一数}{}
\shizi{5-44}{数落}{}
\shizi{5-44}{虎口}{}
\shizi{5-44}{贴心}{}
\shizi{5-44}{才气}{}
\shizi{5-44}{数学}{}
\shizi{5-44}{更小}{}
\shizi{5-44}{数一数二}{}
\shizi{5-44}{更少}{}
\shizi{5-44}{更黑}{}
\shizi{5-44}{更正}{}
\shizi{5-44}{云朵}{}
\shizi{5-44}{一张纸}{}
\shizi{5-44}{纸张}{}
\shizi{5-44}{更多}{}
\shizi{5-44}{更冷}{}
\shizi{5-44}{更加}{}
\shizi{5-44}{花朵}{}
\shizi{5-44}{耳朵}{}
\shizi{5-44}{红纸}{}
\shizi{5-44}{更长}{}
\shizi{5-44}{更新}{}
\shizi{5-44}{变更}{}
\shizi{5-44}{朵朵}{}
\shizi{5-44}{白纸}{}
\shizi{5-44}{纸盒}{}
}

% 宝宝读故事
\SecJuzi{
\shizi{5-44}{ %山羊爷爷的胡子
山羊爷爷有一大把漂亮的胡子,它天天都对它的朋友们说:“谁也没有我这么漂亮的胡子。狮子没有,老虎没有,老狼也没有。只有我山羊爷爷有。哈哈,你们是不是很想学我,也贴上一把胡子呢?”
狮子、老虎、老狼都在想:“是啊,我们是没有山羊爷爷那么一大把漂亮的胡子。”不过,谁也没有想去贴胡子。
可是,山羊爷爷的儿子对山羊爷爷说:
“老爸,你的胡子是很漂亮,我的胡子也很好看。可是,我的儿子小山羊天天哭。他说,他的小朋友们都在笑话他。大家都问他,你这么小,怎么就长了胡子呢?我不知道怎么说,他才不会哭,才能高兴起来?”
山羊爷爷想了又想:“是呀,他这么小怎么也会长胡子呢?”想来想去,山羊爷爷还是没有想明白。}{}
}


\SecDanzi{
\shizi{5-45}{}{}
\shizi{5-45}{}{}
}

% 宝宝读词语
\SecCi{
\shizi{5-45}{}{}
\shizi{5-45}{}{}
}

% 宝宝读短文
\SecJuzi{
\shizi{5-45}{}{}
\shizi{5-45}{}{}
}

\SecDanzi{
\shizi{5-45}{笨}{}
\shizi{5-45}{以}{}
\shizi{5-45}{自}{}
\shizi{5-45}{己}{}
\shizi{5-45}{慢}{}
\shizi{5-45}{难}{}
\shizi{5-45}{练}{}
\shizi{5-45}{每}{}
}

% 宝宝读词语
\SecCi{
\shizi{5-45}{}{}
\shizi{5-45}{}{}
}

% 宝宝读短文
\SecJuzi{
\shizi{5-45}{}{}
\shizi{5-45}{}{}
}


\SecDanzi{
\shizi{5-46}{颗}{}
\shizi{5-46}{样}{}
\shizi{5-46}{因}{}
\shizi{5-46}{为}{}
\shizi{5-46}{离}{}
\shizi{5-46}{近}{}
\shizi{5-46}{象}{}
\shizi{5-46}{船}{}
}

% 宝宝读词语
\SecCi{
\shizi{5-46}{}{}
\shizi{5-46}{}{}
}

% 宝宝读短文
\SecJuzi{
\shizi{5-46}{}{}
\shizi{5-46}{}{}
}


\SecDanzi{
\shizi{5-47}{闪}{}
\shizi{5-47}{金}{}
\shizi{5-47}{美}{}
\shizi{5-47}{丽}{}
\shizi{5-47}{当}{}
\shizi{5-47}{扇}{}
\shizi{5-47}{满}{}
\shizi{5-47}{干}{}
}

% 宝宝读词语
\SecCi{
\shizi{5-47}{}{}
\shizi{5-47}{}{}
}

% 宝宝读短文
\SecJuzi{
\shizi{5-47}{}{}
\shizi{5-47}{}{}
}


\SecDanzi{
\shizi{5-48}{朝}{}
\shizi{5-48}{熊}{}
\shizi{5-48}{娃}{}
\shizi{5-48}{汽}{}
\shizi{5-48}{车}{}
\shizi{5-48}{北}{}
\shizi{5-48}{京}{}
\shizi{5-48}{往}{}
\shizi{5-48}{呜}{}
}

% 宝宝读词语
\SecCi{
\shizi{5-48}{}{}
\shizi{5-48}{}{}
}

% 宝宝读短文
\SecJuzi{
\shizi{5-48}{}{}
\shizi{5-48}{}{}
}




\SecDanzi{
\shizi{5-49}{鹿}{}
\shizi{5-49}{森}{}
\shizi{5-49}{林}{}
\shizi{5-49}{采}{}
\shizi{5-49}{蘑}{}
\shizi{5-49}{菇}{}
\shizi{5-49}{篮}{}
\shizi{5-49}{直}{}
}

% 宝宝读词语
\SecCi{
\shizi{5-49}{}{}
\shizi{5-49}{}{}
}

% 宝宝读短文
\SecJuzi{
\shizi{5-49}{}{}
\shizi{5-49}{}{}
}




\SecDanzi{
\shizi{5-50}{摸}{}
\shizi{5-50}{苔}{}
\shizi{5-50}{所}{}
\shizi{5-50}{科}{}
\shizi{5-50}{灯}{}
\shizi{5-50}{应}{}
\shizi{5-50}{该}{}
\shizi{5-50}{题}{}
}

% 宝宝读词语
\SecCi{
\shizi{5-50}{}{}
\shizi{5-50}{}{}
}

% 宝宝读短文
\SecJuzi{
\shizi{5-50}{}{}
\shizi{5-50}{}{}
}



%宝宝读词语
\SecCi{
\shizi{5-复习7}{}{}
\shizi{5-复习7}{}{}
}


\SecJuzi{
\shizi{5-复习7}{}{}
\shizi{5-复习7}{}{}
}



\end{document}
