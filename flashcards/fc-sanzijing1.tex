%# -*- coding: utf-8 -*-
%!TEX encoding = UTF-8 Unicode
%!TEX TS-program = xelatex
% vim:ts=4:sw=4
%
% 以上设定默认使用 XeLaTex 编译,并指定 Unicode 编码,供 TeXShop 自动识别
\documentclass[avery5371,grid]{flashcards}

\cardfrontstyle[\large\slshape]{headings}
\cardbackstyle{empty}
\cardfrontfoot{三字經}

\newcommand{\doctitle}{三字經}
\newcommand{\docauthor}{}
\newcommand{\dockeywords}{{中文}}
\newcommand{\docsubject}{}

%# -*- coding: utf-8 -*-
%!TEX encoding = UTF-8 Unicode
%!TEX TS-program = xelatex
% vim:ts=4:sw=4
%
% 以上设定默认使用 XeLaTex 编译,并指定 Unicode 编码,供 TeXShop 自动识别


\newcommand\mymainfont{Noto Serif}%{Times New Roman} %{DejaVu Serif}
\newcommand\mymonofont{DejaVu Sans Mono}%{FreeMono} %{WenQuanYi Micro Hei Mono} %{Monaco}
\newcommand\myboldfont{DejaVu Sans Mono} %{WenQuanYi Micro Hei Mono}%{AR PL UKai CN}%{YaHei Consolas Hybrid}%{黑体}%{標楷體}
\newcommand\mysansfont{DejaVu Sans}%{FreeSans}
\newcommand\myitalicfont{DejaVu Serif}%{Times New Roman} %{Garamond}

\newcommand\mycjkboldfont{Microsoft YaHei} %{WenQuanYi Micro Hei Mono}%{Adobe Heiti Std}%{AR PL UKai CN}%{YaHei Consolas Hybrid}%{黑体}%{標楷體}
\newcommand\mycjkitalicfont{全字庫正楷體} %{Adobe Kaiti Std}
\newcommand\mycjkmainfont{WenyueType GutiFangsong (Noncommercial Use)}%{全字庫正楷體} %{Adobe Song Std} %{AR PL UMing CN}%{仿宋}%{宋体}%{新宋体}%{文鼎PL新宋}%
\newcommand\mycjksansfont{MingLiU} %{Adobe Ming Std}
\newcommand\mycjkmonofont{DejaVu Sans YuanTi Mono}%{WenQuanYi Micro Hei Mono}%{AR PL UMing CN}%{WenQuanYi Micro Hei Mono}


%\usepackage[nofonts]{ctex} %adobefonts
\usepackage[adobefonts]{ctex} %adobefonts
%\usepackage[fallback]{xeCJK}


\newCJKfontfamily{\mykaiti}{全字庫正楷體}%{AR PL UKai TW}%{全字庫正楷體}
\newCJKfontfamily{\myfangsong}{WenyueType GutiFangsong (Noncommercial Use)}



\usepackage{ifthen}
\usepackage{ifpdf}
\usepackage{ifxetex}
\usepackage{ifluatex}


\usepackage{color}
\usepackage[rgb]{xcolor}



\xeCJKsetup{AutoFallBack = true} % you have to use this, since fallback won't work as the xeCJK option after the ctex

\PassOptionsToPackage{
    BoldFont,  % 允許粗體
    SlantFont, % 允許斜體
    CJKnumber,
    CJKtextspaces,
    }{xeCJK}
\defaultfontfeatures{Mapping=tex-text} % 如果沒有它,會有一些 tex 特殊字符無法正常使用,比如連字符。

\xeCJKsetup {
    CheckSingle = true,
    AutoFakeBold = false,
AutoFakeSlant = false,
    CJKecglue = {},
    PunctStyle = kaiming,
KaiMingPunct+ = {:;},
}

\PassOptionsToPackage{CJKchecksingle}{xeCJK}
%\defaultCJKfontfeatures{Scale=0.5}
%\LoadClass[c5size,openany,nofonts]{ctexbook}
\XeTeXlinebreaklocale "zh"                      % 重要,使得中文可以正確斷行!
\XeTeXlinebreakskip = 0pt plus 1pt minus 0.1pt  % 给予TeX断行一定自由度
\linespread{1.3}                                % 1.3倍行距

\setCJKmainfont[BoldFont=\mycjkboldfont, ItalicFont=\mycjkitalicfont]{\mycjkmainfont}
\setCJKsansfont{\mycjksansfont}%{Adobe Ming Std} %{AR PL UMing CN} %{Microsoft YaHei}
\setCJKmonofont{\mycjkmonofont}



\ifxetex % xelatex
\else
    %The cmap package is intended to make the PDF files generated by pdflatex "searchable and copyable" in acrobat reader and other compliant PDF viewers.
    \usepackage{cmap}%
\fi
% ============================================
% Check for PDFLaTeX/LaTeX
% ============================================
\newcommand{\outengine}{xetex}
\newif\ifpdf
\ifx\pdfoutput\undefined
  \pdffalse % we are not running PDFLaTeX
  \ifxetex
    \renewcommand{\outengine}{xetex}
  \else
    \renewcommand{\outengine}{dvipdfmx}
  \fi
\else
  \pdfoutput=1 % we are running PDFLaTeX
  \pdftrue
  \usepackage{thumbpdf}
  \renewcommand{\outengine}{pdftex}
  \pdfcompresslevel=9
\fi
\usepackage[\outengine,
    bookmarksnumbered, %dvipdfmx
    %% unicode, %% 不能有unicode选项,否则bookmark会是乱码
    colorlinks=true,
    citecolor=red,
    urlcolor=blue,        % \href{...}{...} external (URL)
    filecolor=red,      % \href{...} local file
    linkcolor=black, % \ref{...} and \pageref{...}
    breaklinks,
    pdftitle={\doctitle},
    pdfauthor={\docauthor},
    pdfsubject={\docsubject},
    pdfkeywords={\dockeywords},
    pdfproducer={Latex with hyperref},
    pdfcreator={pdflatex},
    %%pdfadjustspacing=1,
    pdfborder=1,
    pdfpagemode=UseNone,
    pagebackref,
    bookmarksopen=true]{hyperref}

% --------------------------------------------
% Load graphicx package with pdf if needed 
% --------------------------------------------
\ifxetex    % xelatex
    \usepackage{graphicx}
\else
    \ifpdf
        \usepackage[pdftex]{graphicx}
        \pdfcompresslevel=9
    \else
        \usepackage{graphicx} % \usepackage[dvipdfm]{graphicx}
    \fi
\fi

%\newCJKfontfamily{\yanti}{\mycjkmainfont}
%\newenvironment{shicibody}{%
%\begin{verse}\centering\yanti\large\hspace{13pt}}{\end{verse}}


\usepackage{xstring}

%The baseline-skip should be set to roughly 1.2x the font size.
%\fontsize{size}{baselineskip}
%\fontsize{50}{60}

% 计算所能容纳的字数
\newcommand\shizishow[1]{
\StrLen{#1}[\mystringlen]
%\def\mythresh{1}
    \ifthenelse{\mystringlen < 2}{
        {\fontsize{130}{156} #1}
    }{ \ifthenelse{\mystringlen < 3}{
        {\fontsize{110}{122} #1}%{\fontsize{100}{120} #1}
    }{ \ifthenelse{\mystringlen < 4}{
        {\fontsize{78}{94} #1}%{\fontsize{90}{108} #1}
    }{ \ifthenelse{\mystringlen < 5}{
        {\fontsize{55}{66} #1}
    }{ \ifthenelse{\mystringlen < 7}{
        {\fontsize{70}{84} #1}
    }{ \ifthenelse{\mystringlen < 9}{
        {\fontsize{55}{66} #1}
    }{ \ifthenelse{\mystringlen < 16}{
        {\zihao{0} #1} %5x4=20
    }{ \ifthenelse{\mystringlen < 49}{
        {\zihao{1} #1} % 8x6=48
    }{ \ifthenelse{\mystringlen < 55}{
        {\Huge #1} % 9x6=54
    }{ \ifthenelse{\mystringlen < 61}{
        {\zihao{2} #1} % 10x6=60
    }{ \ifthenelse{\mystringlen < 78}{
        {\huge #1} % 11x7=77
    }{ \ifthenelse{\mystringlen < 105}{
        {\LARGE #1} % 13x8=104
    }{ \ifthenelse{\mystringlen < 113}{
        {\zihao{3} #1} % 14x8=112
    }{ \ifthenelse{\mystringlen < 129}{
        {\Large #1} % 16x8=128 %{\zihao{4} #1} % 16x8=128
    }{ \ifthenelse{\mystringlen < 153}{
        {\large #1} % 19x8=152, or 19x9=171
    }{ \ifthenelse{\mystringlen < 198}{
        {\zihao{5} #1} % 22x9=198
    }{ \ifthenelse{\mystringlen < 207}{
        {\normalsize #1} % 23x9=207
    }{ \ifthenelse{\mystringlen < 234}{
        {\small #1} % 26x9=234
    }{ \ifthenelse{\mystringlen < 261}{
        {\footnotesize #1} % 29x9=261
    }{ \ifthenelse{\mystringlen < 279}{
        {\zihao{6} #1} % 31x9=279
    }{ \ifthenelse{\mystringlen < 297}{
        {\scriptsize #1} % 33x9=297
    }{ \ifthenelse{\mystringlen < 378}{
        {\zihao{7} #1} % 42x9=378
    }{
        {\tiny #1} % 46x9=414 %{\zihao{8} #1} % 46x9=414
    }}}}}}}}}}}}}}}}}}}}}}
}

% 计算包含拼音时所能容纳的字数
\newcommand\shizipy[1]{
\StrLen{#1}[\mystringlen]
\xpinyin*{
    \ifthenelse{\mystringlen < 6}{
        {\zihao{0} #1} %5x4=20
    }{ \ifthenelse{\mystringlen < 18}{
        {\zihao{1} #1} % 8x6=48
    }{ \ifthenelse{\mystringlen < 22}{
        {\Huge #1} % 9x6=54
    }{ \ifthenelse{\mystringlen < 31}{
        {\zihao{2} #1} % 10x6=60
    }{ \ifthenelse{\mystringlen < 34}{
        {\huge #1} % 11x7=77
    }{ \ifthenelse{\mystringlen < 53}{
        {\LARGE #1} % 13x8=104
    }{ \ifthenelse{\mystringlen < 57}{
        {\zihao{3} #1} % 14x8=112
    }{ \ifthenelse{\mystringlen < 65}{
        {\Large #1} % 16x8=128 %{\zihao{4} #1} % 16x8=128
    }{ \ifthenelse{\mystringlen < 77}{
        {\large #1} % 19x8=152, or 19x9=171
    }{ \ifthenelse{\mystringlen < 89}{
        {\zihao{5} #1} % 22x9=198
    }{ \ifthenelse{\mystringlen < 93}{
        {\normalsize #1} % 23x9=207
    }{ \ifthenelse{\mystringlen < 105}{
        {\small #1} % 26x9=234
    }{ \ifthenelse{\mystringlen < 117}{
        {\footnotesize #1} % 29x9=261
    }{ \ifthenelse{\mystringlen < 125}{
        {\zihao{6} #1} % 31x9=279
    }{ \ifthenelse{\mystringlen < 133}{
        {\scriptsize #1} % 33x9=297
    }{ \ifthenelse{\mystringlen < 169}{
        {\zihao{7} #1} % 42x9=378
    }{
        {\tiny #1} % 46x9=414 %{\zihao{8} #1} % 46x9=414
    }}}}}}}}}}}}}}}}
}
}


% 计算在comment中(三字经)能容纳的字数
\newcommand\sanzicomments[1]{
\StrLen{#1}[\mystringlen]
%\def\mythresh{1}
    \ifthenelse{\mystringlen < 7}{
        {\zihao{0} #1} %5x4=20
    }{ \ifthenelse{\mystringlen < 22}{
        {\zihao{1} #1} % 8x6=48
    }{ \ifthenelse{\mystringlen < 25}{
        {\Huge #1} % 9x6=54
    }{ \ifthenelse{\mystringlen < 38}{
        {\zihao{2} #1} % 10x6=60
    }{ \ifthenelse{\mystringlen < 42}{ %
        {\huge #1} % 11x7=77
    }{ \ifthenelse{\mystringlen < 63}{
        {\LARGE #1} % 13x8=104
    }{ \ifthenelse{\mystringlen < 64}{
        {\zihao{3} #1} % 14x8=112
    }{ \ifthenelse{\mystringlen < 87}{
        {\Large #1} % 16x8=128 %{\zihao{4} #1} % 16x8=128
    }{ \ifthenelse{\mystringlen < 125}{
        {\large #1} % 19x8=152, or 19x9=171
    }{ \ifthenelse{\mystringlen < 173}{
        {\zihao{5} #1} % 22x9=198
    }{ \ifthenelse{\mystringlen < 205}{
        {\normalsize #1} % 23x9=207
    }{ \ifthenelse{\mystringlen < 232}{
        {\small #1} % 26x9=234
    }{ \ifthenelse{\mystringlen < 259}{
        {\footnotesize #1} % 29x9=261
    }{ \ifthenelse{\mystringlen < 277}{
        {\zihao{6} #1} % 31x9=279
    }{ \ifthenelse{\mystringlen < 295}{
        {\scriptsize #1} % 33x9=297
    }{ \ifthenelse{\mystringlen < 376}{
        {\zihao{7} #1} % 42x9=378
    }{
        {\tiny #1} % 46x9=414 %{\zihao{8} #1} % 46x9=414
    }}}}}}}}}}}}}}}}
}



% 诗词
\newcommand\shici[5]{
\begin{flashcard}[#1 -- #2 (#3) #4]{%
{\mykaiti \fontsize{20}{24} {#5}} %\\
}
\vspace*{\stretch{1}}
\begin{center}

{\mykaiti \zihao{1} {#2}}\\ \vspace*{\stretch{.5}}
{\large(#3)}\\ \vspace*{\stretch{.25}}
{\LARGE #4}

\end{center}
\vspace*{\stretch{1}}
\end{flashcard}
}



% 识字

%\usepackage[overlap,CJK]{ruby}
\usepackage{xpinyin}

% 如果字数多,则略写
\newcommand\shizititle[2]{
  \StrLen{(#1) #2}[\mystringlen]%
  \ifthenelse{\mystringlen > 24}{
    \StrLeft{(#1) #2}{20}......\StrRight{(#1) #2}{4}
  }{
  (#1) #2
  }
}
\newcommand\shizi[3]{
\begin{flashcard}[\shizititle{#1}{#2}]{ %
{\mykaiti \shizipy{#2}}

{\myfangsong \normalsize #3} %
} %
\vspace*{\stretch{1}}
\begin{center}
%ēéěè
%\ruby{莉}{li}
%\xpinyin*{床前明月光,疑是地上霜。举头望明月,低头思故乡。}
%\begin{pinyinscope}
%床前明月光,疑是地上霜。举头望明月,低头思故乡。
%\end{pinyinscope}

{\mykaiti \shizishow{#2}}

\end{center}
\vspace*{\stretch{1}}
\end{flashcard}
}


\newcounter{fcsanzi}\setcounter{fcsanzi}{0}

% 三字经
\newcommand\sanzijing[3]{
\stepcounter{fcsanzi}
\begin{flashcard}[]{
\mykaiti \fontsize{26}{\baselineskip}\selectfont \xpinyin*{#1}
}
\vspace*{\stretch{1}}
%\begin{center}
%\normalsize
{
\myfangsong
\sanzicomments{
【解释】 #2

〖解读〗 #3
}
}
%\end{center}
\vspace*{\stretch{1}}
\end{flashcard}
}





\newcommand\docshowcopyright{
\begin{flashcard}[Copyright \& License]{Copyright \copyright \, 2014 Yunhui Fu\\
Some rights reserved.}
\vspace*{\stretch{1}}
These flashcards and the accompanying \LaTeX \, source code are licensed
under a Creative Commons Attribution--NonCommercial--ShareAlike 2.5 License.  
For more information, see creativecommons.org.  You can contact the author at:
\begin{center}
\begin{small}\tt yhfudev at gmail com\end{small}

\medskip
File last updated on \today, \\
at \currenttime
\end{center}
\vspace*{\stretch{1}}
\end{flashcard}

\begin{flashcard}[版权申明]{版权所有 \copyright \, 2014 Yunhui Fu\\
有些版权保留.}
\vspace*{\stretch{1}}
闪卡及其 \LaTeX \, 源代码在
署名-相同方式共享 2.5 下保护.
参见 creativecommons.org.  你可以联系作者:
\begin{center}
\begin{small}\tt yhfudev at gmail com\end{small}

\medskip
文件最近更新: \today, \\
\currenttime
\end{center}
\vspace*{\stretch{1}}
\end{flashcard}
}






\newcommand\docshowtitle[3]{
\begin{flashcard}[]{ %
{\mykaiti \Huge{#1}}

{\myfangsong \normalsize #2} %
} %
\vspace*{\stretch{1}}
\begin{center}

{\mykaiti \shizishow{#3}}

\end{center}
\vspace*{\stretch{1}}
\end{flashcard}
}







\usepackage{datetime}

\begin{document}

\docshowcopyright
\docshowtitle{\doctitle}{\docauthor}{%
使用双面打印,然后按线剪下。
}


% http://ccigf01.blogspot.com/search/label/%E4%B8%89%E5%AD%97%E7%B6%93-01
\sanzijing{人之初、性本善、 \\性相近、習相遠、} % 原文
{人初生之時,本性都是善良的。善良的本性彼此都很接近
,後來因為生活和學習環境的不同,差異越來越大。} % 解释
{(失之毫釐,差以千里)} % 解读

\sanzijing{苟不教、性乃遷、 \\教之道、貴以專、} % 原文
{如果不及早接受良好的教育,善良的本性就會隨環境的
影響而改變,所謂先入為主,不可不慎!至於教育方法
,應注重在使孩子專心,有定力。課業的選擇,要以專
精為主,不要希求廣博。} % 解释
{(據心理學家研究發現:兒童
與青少年心智的發展,十三歲以前著重在記憶,之後理
解力逐漸成長,故十三歲以前應注重記憶念誦為主,理
解次之,不可本末倒置。)} % 解读

\sanzijing{昔孟母、擇鄰處、 \\子不學、斷機杼、} % 原文
{古時候孟子的母親,為了尋找一個對孟子有益的教育環境
,不辭辛勞搬了三次家。(從墳場附近搬到市場邊,再三
遷至學校旁。)有一次孟子不用功,逃學回家,孟母當著
他的面將織了一半的布匹剪斷,並且告誡他說:「求學的
道理,就像織布一樣,必須將紗線一條一條織上去,經過
持續不斷的努力,積絲才能成寸,積寸才能成尺,最後才
能織成一匹完整有用的布;讀書也是一樣,要努力用功,
並且持之以恆,經過長時間的累積,才能有成就。否則就
像織布半途而廢一樣,一旦前功盡棄就毫無用處了。} % 解释
{} % 解读

\sanzijing{竇燕山、有義方、 \\教五子、名俱揚、} % 原文
{五代時,有一位竇禹鈞(又稱竇燕山),遵照聖賢教誨
的義理來教育子女,因此五個兒子都很有成就,都能光
耀家門。} % 解释
{} % 解读

\sanzijing{養不教、父之過、 \\教不嚴、師之惰、} % 原文
{生育子女,若只知道養活他們,而不去教育,那是作父母
的失職。老師教導學生,不只是知識、技藝的傳授,更重
要的是教導學生做人處世的道理,使學生能夠與人相處融
洽,做事有方法,活得健康愉快有意義。因此對於學生的
要求一定要認真嚴格,不能偷懶怠惰,才能教出好學生。} % 解释
{(嚴師出高徒,嚴是認真,一絲不苟的態度。要教出好孩
子,必須父母與老師雙方面配合,也就是家庭教育和學校
教育共同努力。)} % 解读




\sanzijing{子不學、非所宜、 \\幼不學、老何為、} % 原文
{為人子女如果不用心學習,是不對的!年紀小的時候
,不肯努力、用功學習,等到年紀大了,還能有什麼
作為呢?} % 解释
{(少壯不努力,老大徒傷悲。)} % 解读



\sanzijing{玉不琢、不成器、 \\人不學、不知義、} % 原文
{一塊玉石,如果不去雕琢,就不能成為有用的器具;
人也是一樣,如果不透過學習,就無法明白做人處事
的道理,不知道那些事合不合乎義理,應不應該做?} % 解释
{} % 解读


\sanzijing{為人子、方少時、 \\親師友、習禮儀、} % 原文
{作子弟的,要趁著年少的時候親近良師、結交益友,
好好學習待人、處事、應對、進退的禮儀。} % 解释
{} % 解读



\sanzijing{香九齡、能溫席、 \\孝於親、所當執、} % 原文
{黃香是東漢江夏人,他在九歲的時候,就懂得孝順父母
,夏天天氣熱的時候,他就用扇子先將父母的床鋪搧涼
,再請父母就寢;冬天天冷的時候,他就先將父母親的
被子睡溫暖,再請父母安睡。類似這種孝順的行為,是
每一個為人子女所應當盡的本分。} % 解释
{} % 解读



\sanzijing{融四歲、能讓梨、 \\弟於長、宜先知、} % 原文
{東漢末年的孔融,才四歲的時候,就曉得禮讓兄長,
將大的梨子讓兄長吃,自己選擇較小的。這種尊敬兄
長、友愛兄弟的美德,應當及早教育培養。} % 解释
{} % 解读





\sanzijing{首孝弟、次見聞、 \\知某數、識某文、} % 原文
{做人第一重要的是孝順父母,友愛兄弟,其次才是增廣
見聞學習知識,明白數字的變化,並研讀古聖先賢的文
章,來修養自己。} % 解释
{} % 解读



\sanzijing{一而十、十而百、 \\百而千、千而萬、} % 原文
{一是數字的開始,十個十是一百,十個一百是一千,
十個千是一萬,如此累積上去,可以無窮無盡。} % 解释
{} % 解读



\sanzijing{三才者、天地人、 \\三光者、日月星、} % 原文
{古人以為構成生命現象與生命意義的基本要素是:天、地
、人;「天」是指萬物賴以生存的空間,包括日月星辰運
轉不息,四季更替而不亂。晝夜寒暑都有一定的次序;
「地」是指萬物藉以生長的地理條件和各種物產;「人」
是萬物之靈,要順天地化育萬物。三種光明的來源是指
日、月、星。} % 解释
{} % 解读



\sanzijing{三綱者、君臣義、 \\父子親、夫婦順、} % 原文
{維持人與人之間最重要的三種倫常關係,就是:君臣之
間有道義,父子之間有親情,夫妻之間能相互尊重和睦
相處。} % 解释
{(君義臣忠,父慈子孝,夫義婦順。)} % 解读



\sanzijing{曰春夏、曰秋冬、 \\此四時、運不窮、} % 原文
{一年之中春、夏、秋、冬四季各有特色,循環運轉,永不
止息。(春耕、夏耘、秋收、冬藏)四季變化是因為地球
繞著太陽公轉的關係。} % 解释
{} % 解读



\sanzijing{曰南北、曰西東、 \\此四方、應乎中、} % 原文
{東、南、西、北這四個方向,都以中央為準互相對應。} % 解释
{(如果沒有中心,就沒有東西南北,因此中心移動,方
向也跟著改變。)} % 解读



\sanzijing{曰水火、木金土、 \\此五行、本乎數、} % 原文
{古人以木火土金水,為構成物質的五種基本特性,稱為五
行,並將這五行的變化以數學的原理加以歸納,發現它們
之間有相生相剋的關係,如水生木、木生火、火生土、土
生金是相生的關係;金剋木、木剋土、土剋水、水剋火、
火剋金是相剋的關係。} % 解释
{} % 解读


\sanzijing{十干者、甲至癸、 \\十二支、子至亥、} % 原文
{十干是指:甲乙丙丁戊己庚辛壬癸,天干有十個,地支
是指:子丑寅卯辰巳午未申酉戍亥,地支有十二個,都
是古人計算時間的符號;地支用來計算每日的時間,如
:子時是指凌晨11時至 1時,丑時是指 1時至 3時,午時
是中午11至13時。天干與地支互相配合用來計算年,如
甲子年、乙丑年等。} % 解释
{(六十年為一甲子)} % 解读


\sanzijing{曰黃道、日所躔、 \\曰赤道、當中權、} % 原文
{黃道是太陽在太空中所運行的軌道,赤道是地球的中線,
將地球分為南半球和北半球。} % 解释
{} % 解读


\sanzijing{赤道下、溫暖極、 \\我中華、在東北、} % 原文
{赤道附近為熱帶,溫度極為炎熱,我們的國家,位置在
北半球的東邊,因為上地面積廣大,所以包含了熱帶、
溫帶、和寒帶。} % 解释
{} % 解读


\sanzijing{曰江河、曰淮濟、 \\此四瀆、水之紀、} % 原文
{我們的國家有美麗的山河,河川之中長江是第一大河,
黃河是第二大河,還有淮河和濟水,這四條河最後 都向
東流入大海。} % 解释
{} % 解读


\sanzijing{曰岱華、嵩恆衡、 \\此五岳、山之名、} % 原文
{東嶽泰山,西嶽華山,中嶽嵩山,北嶽恒山,南嶽衡
山 。這五座山稱為五岳,是中國的五大名山。} % 解释
{} % 解读


\sanzijing{曰士農、曰工商、 \\此四民、國之良、} % 原文
{士是讀書人,士農工商這四種身份的人民,是組成社會、建
立國家的基本份子。} % 解释
{} % 解读





\sanzijing{曰仁義、禮智信、 \\此五常、不容紊、} % 原文
{仁、義、禮、智、信,維繫著人與人之間的關係,這
五種道理永遠不變,必須遵守不容許混淆。} % 解释
{} % 解读

\sanzijing{地所生、有草木、 \\此植物、遍水陸、} % 原文
{大地所生長的生物,種類繁多,像草木是屬於植物,
遍及陸上及水上。} % 解释
{} % 解读

\sanzijing{有蟲魚、有鳥獸、 \\此動物、能飛走、} % 原文
{至於動物有蟲、魚、鳥、獸,它們有的能在空中飛翔,
有的是行走在陸地上,還有的生活在水中。} % 解释
{} % 解读

\sanzijing{稻梁菽、麥黍稷、 \\此六穀、人所食、} % 原文
{稻米、高梁、黃豆、麥、黍(粘米)、稷六種穀物是供
人類所食用的主食;中國地大物博,各地氣候、風俗、
民情不同,因此主食各不相同。} % 解释
{} % 解读

\sanzijing{馬牛羊、雞犬豕、 \\此六畜、人所飼、} % 原文
{馬牛羊雞狗豬,這六種動物是人類所畜養的,各有其
貢獻。} % 解释
{} % 解读

\sanzijing{曰喜怒、曰哀懼、 \\愛惡欲、七情具、} % 原文
{喜悅快樂、生氣、憂傷、害怕不安、愛惜眷戀、憎恨
討厭、及想要擁有的慾望,是人人都具備的七種情緒
。} % 解释
{(有智慧、有修養的人能適當的調節控制,不受情
緒所左右,做情緒的主人,不作情緒的奴隸。)人類所畜養的,各有其
貢獻。} % 解读





\sanzijing{青赤黃、及黑白、 \\此五色、目所識、} % 原文
{青、紅、黃、黑、白,是我國古代所定的五種顏色,
稱為五色,我們很容易就能用眼睛辨別出來。} % 解释
{} % 解读

\sanzijing{酸苦甘、及辛鹹、 \\此五味、口所含、} % 原文
{易解:酸、苦、甜(甘)、以及辣(辛)和鹹這五種
味道,是食物中所包含的五種味道。} % 解释
{} % 解读

\sanzijing{羶焦香、及腥朽、 \\比五臭、鼻所嗅、} % 原文
{羊臊味(羶)、燒焦味、香味、以及魚腥味、腐爛的
臭味,這五種氣味,是我們的鼻子所聞到的五種味道
。} % 解释
{} % 解读

\sanzijing{匏土革、木石金、 \\絲與竹、乃八音、} % 原文
{匏瓜與黏土古人作成吹奏的樂器(如笙、壎),革是指
牛皮,可以製成樂器稱為鼓,有振奮人心的作用,如晉
鼓、腰鼓、博浪鼓等。木製的樂器有木魚、梆子、拍板
、祝和敔。(常用於祭孔大典中),石類樂器有磬(石
磬),金屬鑄成的樂器有鐘、鑼、鈸、鐃等。絲類樂器
是因為中國古代用蠶絲作弦,故稱為絲類,有琴、瑟、
箏以及後來的胡琴、琵琶、小提琴、吉他等都是。竹製
樂器是用竹管穿孔而製成,主要分為笛(橫吹)和簫(
直吹)。以上這八種樂器可作為八音的代表,「音樂」
可以調和身心,它和「禮」是相輔相成的。} % 解释
{} % 解读

\sanzijing{曰平上、曰去入、 \\此四聲、宜調協、} % 原文
{平、上、去、入是古時候的四聲,講話時咬字發音應該
正確,才能讓人聽得清楚明白。平聲即是現在國音中的
第一和第二聲,上聲即是國音的第三聲,去聲即是國音
的第四聲,入聲音調短促而急,分佈在國音聲調中。} % 解释
{} % 解读

\sanzijing{高曾祖、父而身、 \\身而子、子而孫、} % 原文
{高曾祖代表了三代,即是高祖、曾祖、祖父,然後是父親
再來是自己本身。往下是兒子、孫子、曾孫、玄孫。} % 解释
{} % 解读







\sanzijing{自子孫、至玄曾、 \\乃九族、人之倫、} % 原文
{往下是兒子、孫子、曾孫、玄孫。一共九代親族稱為九
族,包含自身以上的上四代及下四代,是我們的直系血
親,和自己關係最為密切。是家族中長幼尊卑基本的倫
常關係。} % 解释
{} % 解读


\sanzijing{父子恩、夫婦從、 \\兄則友、弟則恭、} % 原文
{以下是講五倫,五倫彼此之間都有互相對待的原則。
父慈子孝,父親慈祥恩愛,子女孝順,夫義婦順從,
夫妻之間應該互相尊重體諒。至於兄弟姊妹之間,作
兄長姊姊的應該愛護弟妹,弟妹也應該恭敬兄長姊姊
。} % 解释
{} % 解读


\sanzijing{長幼序、友與朋、 \\君則敬、臣則忠、} % 原文
{長幼之間要有倫常秩序,朋友相處也要誠實互信真心的
交往,領導者對部屬要尊重,部屬對長官應忠於職守認
真做事,各盡本分。} % 解释
{} % 解读


\sanzijing{此十義、人所同、 \\當順敘、勿違背、} % 原文
{從父子恩到臣則忠,這十條義理,每一個人都要遵守
奉行,要了解親疏關係,注意先後順序不可以違背。} % 解释
{} % 解读


\sanzijing{斬齊衰、大小幼、 \\至緦麻、五服終、} % 原文
{古時候父母、祖父母、兄弟、伯叔、外祖父與表兄弟等
親人,過世時所穿的喪服都有一定的禮節,不可混亂。} % 解释
{} % 解读

\sanzijing{禮樂射、卸書數、 \\古六藝、今不具、} % 原文
{學習禮節儀規、音樂、射箭、駕馭馬車、書法、數學,
是古人教導子弟必備的六種技能,稱為六藝,現在的學
生已經不具備這些才能了。} % 解释
{} % 解读





\sanzijing{惟書學、人共遵、 \\既識字、講說文、} % 原文
{只有練習書法寫字,大家仍然共同遵守。已經認識字之
後,接著就要講解學習文字的構造和它的意義。} % 解释
{} % 解读

\sanzijing{有古文、大小篆、 \\隸草繼、不可亂、} % 原文
{中國文字源遠流長,有他獨特的意義,較早有甲骨文、
鐘鼎文,再來是大篆、小篆,繼之有隸書、草書、這些
都要分清楚不可混亂。} % 解释
{} % 解读

\sanzijing{若廣學、懼其繁、 \\但略說、能知原、} % 原文
{天地之間的學問廣大無邊,如果都想要學,恐怕會無從
學起,不如選擇一門深入,長時薰修,日積月累功夫精
深,自然水到渠成成就非凡,其餘的只須知其概要,瞭
解來源就可以了。} % 解释
{} % 解读

\sanzijing{凡訓蒙、需講究、 \\詳訓詁、明句讀、} % 原文
{凡是教導初學的學童,一定要講究教學方法,義理講解
要詳細,並且考察事實說明清楚。對於文字章句的讀法
,與如何分段的方法都要明白。} % 解释
{} % 解读

\sanzijing{為學者、必有初、 \\小學終、至四書、} % 原文
{讀書求學,必須有一個好的開始,才能奠定良好的基礎
,應先熟悉宋朝朱熹所著的小學這本書,學習灑掃應對
及六藝等,再來深究四書當中修齊治平的大學問。} % 解释
{} % 解读

\sanzijing{論語者、二十篇、 \\群弟子、記善言、} % 原文
{論語是孔老夫子教學傳道的記錄,一共有二十篇,內容
是孔子的學生記載聖人的言行,談論為人、處世與為政
行仁的言論,包含了夫子與學生之間的對話,或學生與
學生之間相互問答的記錄,十分難能可貴。} % 解释
{} % 解读





\sanzijing{孟子者、七篇止、 \\講道德、說仁義、} % 原文
{孟子這本書共有七篇,是學生萬章與公孫丑等記錄老師
言行的書。內容都是講述道德仁義的事,如崇尚王道,
排斥霸道,闡明性善闢斥邪說等,是從政治國的典範。} % 解释
{} % 解读

\sanzijing{作中庸、乃孔伋、 \\中不偏、庸不易、} % 原文
{中庸這一本書是孔夫子的孫子孔伋所作中是不偏不倚
,庸是不易,不變的意思,中庸所說是不偏於一方,
永不改變的天下至理。} % 解释
{} % 解读

\sanzijing{作大學、乃曾子、 \\自修齊、至平治、} % 原文
{大學這一本書是孔夫子的學生曾子所寫,本書共有十章
,內容闡述一個人從修身齊家到治國平天下的大道理。} % 解释
{} % 解读

\sanzijing{孝經通、四書熟、 \\如六經、始可讀、} % 原文
{孝經共十八章,是一部闡明孝道的書。俗話說:「百善孝
為先」因此古人研究學問,首先讀孝經,要把孝經這一部
書的道理,都融會貫通,再讀四書,明白做人處世的道理
,並且有了學問的基礎,然後才能研究六經這些深奧的典
籍。} % 解释
{} % 解读

\sanzijing{詩書易、禮春秋、 \\號六經、當講求、} % 原文
{詩經、書經、易經、禮記、周禮、春秋合稱為六經,凡
是有志於讀書的人,都應當仔細研究其中的道理。} % 解释
{} % 解读

\sanzijing{有連山、有歸藏、 \\有周易、三易詳、} % 原文
{古時候的易經有三種版本,分別為連山、歸藏、和周易,
共稱為三易,連山和歸藏已經失傳了,如今流傳下來的只
有周易一種,孔老夫子曾經審訂過,書中的理論比較詳盡
易懂。} % 解释
{} % 解读






\sanzijing{有典謨、有訓誥、 \\有誓命、書之奧、} % 原文
{書經之中包含典謨、訓誥、誓、命等六篇文書,典:常也
。堯典、舜典是帝王不易的常道。謨是大臣獻上的計策如
大禹謨。訓:誨也,是大臣對君主的進諫,如伊訓。誥是
君主發佈的命令,如昭告、酒誥等。誓:信也,是指君主
出征時宣誓的文告,如甘誓、秦誓等。命是君主所下達的
命令,這些都是書經中奧妙之所在。} % 解释
{} % 解读

\sanzijing{我周公、作周禮、 \\著六官、存治體、} % 原文
{周公制定周禮,分官設職建立國家的政治制度,六官即六卿,包含了天官:吏部大塚宰,地官:戶部大司徒,春官:禮部大宗伯,夏官:兵部大司馬,秋官:刑部大司寇,冬官:工部大司空,分屬於天子之下,各司其職管理國家大事,為後世保存了良好的政治典範。} % 解释
{} % 解读

\sanzijing{大小戴、註禮記、 \\述聖言、禮樂備、} % 原文
{漢朝時有兩位著名的儒者,述說聖人的言論,大戴戴德將
禮記刪訂為85篇,小戴戴聖則刪訂為46篇(即為現今留存
者,加上後人增補3篇合計為49篇)。其內容完整的保存
了古聖先賢的言論,包含各種禮節、五分十二律等音樂都
十分完備。} % 解释
{} % 解读

\sanzijing{曰國風、曰雅頌、 \\號四詩、當諷詠、} % 原文
{詩經按體裁可分為國風、大雅、小雅、頌四種文體稱作四詩
,古時候詩歌一體有詩必有歌,國風是採集自諸侯各國的民
俗歌謠。雅者正也,是正式場合演唱的歌。大雅是諸侯朝覲
天子所用的詩歌,小雅是天子宴享賓客所用的詩歌。頌則是
宗廟祭祀時所使用的樂歌,有周頌、魯頌、商頌三種。} % 解释
{} % 解读

\sanzijing{詩既仁、春秋作、 \\寓褒貶、別善惡、} % 原文
{周平王向東遷都洛陽後(東周),周天子衰落不能號令天
下時,作詩的風氣就逐漸沒落消失了。於是孔老夫子寫出
春秋這一本書,文字雖然簡約,意義十分深遠,詳記魯隱
公到魯哀公二百四十年間的歷史,用來褒揚善行好事,貶
抑惡行壞事,希望能藉此提醒世人分辨忠奸善惡,更期盼
當政者知所警惕,當時很受到重視,對時局有很重要的影
響。} % 解释
{} % 解读

\sanzijing{三傳者、有公羊、 \\有左氏、有穀梁、} % 原文
{傳是解釋「經」的書,這三本傳都是針對「春秋」作註解
,有魯國公羊高寫的公羊傳,有與孔夫子同時代的左丘明
所寫的左傳,還有漢朝穀梁赤所著的穀梁傳。其中左傳使
用編年紀事的體裁為春秋作註解,最令人稱道,舉凡天子
諸侯之事,兵革禮樂之文,興衰存滅之因,都記載得很詳
盡,是研讀春秋一書最佳的選擇。} % 解释
{} % 解读







\sanzijing{經既明、方讀子、 \\撮其要、記其事、} % 原文
{四書和六經的要旨都明白之後,才可以讀諸子百家的書,
如老子、莊子、荀子等,但是由於諸子百家書籍太多,卷
帙浩繁,其內容也有可取之處,但並非完全正確,只要選
取其中對我們的德行學問有幫助的精華來讀,就可以了。} % 解释
{} % 解读

\sanzijing{五子者、有荀揚、 \\文中子、及老莊、} % 原文
{諸子書籍繁多,故有諸子百家之稱,其中最重要的有荀子
、揚子、文中子、老子及莊子。} % 解释
{} % 解读

\sanzijing{經子通、讀諸史、 \\考世系、知終始、} % 原文
{經書和子書融會貫通之後,就可以開始研讀各種史書,史
書是記載一國興亡的事,要從中考察歷代王朝傳承的世系
,明白各國政治上的利弊得失,和治亂興亡的原因,給自
己一個警惕。} % 解释
{} % 解读

\sanzijing{自羲農、至黃帝、 \\號三皇、居上世、} % 原文
{從伏羲、神農到黃帝,這三位上古時代的君主,後人尊
稱他們為「三皇」。史前時代,沒有文字記錄,所以太
古的事無從考察。} % 解释
{} % 解读

\sanzijing{唐有虞、號二帝、 \\相揖遜、稱盛世、} % 原文
{傳至唐堯和虞舜合稱二帝,他們都把國家當作公器,傳賢不
傳子,把帝位禪讓給賢能的人,因為沒有一點私心,造就了
一番太平盛世。} % 解释
{} % 解读

\sanzijing{夏有禹、商有易、 \\周文武、稱三王、} % 原文
{夏朝的第一位君主是大禹,他治理水患,疏通九河,把洪水
引導入大海,老百姓都很擁戴他,因此舜帝把王位傳給他,
夏禹、商湯、周文王、周武王因為賢能被尊稱為
「三代的聖王」。} % 解释
{} % 解读







\sanzijing{夏傳子、家天下、 \\四百載、遷夏社、} % 原文
{後來夏禹準備把帝位傳給益,因為人民懷念禹治水的功績,
加上兒子啟又非常賢能,因此諸侯擁戴啟為天子。中國從
此變成家天下。夏朝傳了四百年就改朝換代了。} % 解释
{} % 解读

\sanzijing{湯伐夏、國號商、 \\六百載、至紂亡、} % 原文
{夏桀在位時因為暴虐無道,成湯起而討伐,建立了新王
朝國號商。傳了六百多年,到紂王時就滅亡了。} % 解释
{} % 解读

\sanzijing{周武王、始誅紂、 \\八百載、最長久、} % 原文
{周文王行仁政,諸侯都來歸附,直到周武王時因為紂王荒淫
無道。才聯合諸侯討伐紂王,建立周朝,周朝傳了八百七十
四年,是我國歷史上年代最長久的王朝。} % 解释
{} % 解读

\sanzijing{周轍東、王綱墜、 \\逞干戈、尚游說、} % 原文
{周武王在鎬京建立國都,史稱西周,到周平王遷都洛陽之後
改稱東周,東遷之後王室的威望低落,王室的綱紀和政治制
度逐漸瓦解,諸侯各自為王,為了擴張勢力彼此爭執干戈相
向,天下變得紛亂不堪。一些謀士與投機份子趁機崛起,周
遊列國擔任說客謀取功名,在各國之間進行遊說,有的主張
連橫,有的提倡合縱,從此兵連禍結,天下蒼生不得安寧。} % 解释
{} % 解读

\sanzijing{始春秋、終戰國、 \\五霸強、七雄出、} % 原文
{自周平王東遷開始稱為春秋時期,共二百二十四年,自韓
趙魏三家分晉後,史家稱戰國時期。春秋時期產生了五位
霸主,依序為齊桓公、宋襄公、晉文公、秦穆公、楚莊王
,他們雖然標榜著崇高的理想,實際上仍依賴著武力稱霸
一時,甚至連周王室都被諸侯滅亡了。進入戰國時期後產
生了齊、楚、燕、趙、韓、魏、秦等七個強國。七雄間彼
此戰爭,弱肉強食殺伐不斷民不聊生。} % 解释
{} % 解读

\sanzijing{嬴秦氏、始兼併、 \\傳二世、楚漢爭、} % 原文
{秦始王姓嬴名政,採用張儀的連橫外交及遠交近攻等策略
,將六國個個擊破,兼併六國的領土,統一天下自稱秦始
皇。只可惜不行仁政,暴虐無道民不聊生,傳到第二代胡
亥時就被抗暴隊伍項羽和劉邦推翻了。楚霸王項羽和漢王
劉邦相爭,兩軍交戰七十多回合,最後項羽兵敗自殺,天
下又成統一的局面。} % 解释
{} % 解读








\sanzijing{高祖興、漢業建、 \\至孝平、王莽篡、} % 原文
{漢高祖劉邦興起,建立漢朝的基業,這是歷史上第一位平民
皇帝。傳到第十一代漢平帝時,被外戚王莽奪取了帝位。王
莽篡漢後,改國號為新。} % 解释
{} % 解读

\sanzijing{光武興、為東漢、 \\四百年、終於獻、} % 原文
{漢光武帝劉秀,復興漢室,推翻王莽,在洛陽建都,稱為
東漢。(王莽以前的時代稱為西漢)兩漢共傳了四百多年
,到漢獻帝時,被曹操之子曹丕所廢。} % 解释
{} % 解读

\sanzijing{魏蜀吳、爭漢鼎、 \\號三國、迄兩晉、} % 原文
{魏(曹操、曹丕),蜀(劉備),吳(孫權)互爭漢家天下
,歷史上稱為三國時代,一直到晉朝(司馬炎)繼起,滅了
三國,才結束紛亂的局面。又因為五胡亂華,把國都遷到江
南建康稱為東晉。} % 解释
{} % 解读

\sanzijing{宋齊繼、梁陳承、 \\為南朝、都金陵、} % 原文
{晉朝傳了一百多年之後至晉安帝時被劉裕所篡,從此進入南北
朝的時代。北方被外族所統治,南方偏安局面下的朝廷,稱為
南朝。為了與其他的朝代有所區別,史學家便將這四國的國號
加上「南」字,分別為南宋、南齊、南梁、南陳,皆在金陵
(南京)建都。} % 解释
{} % 解读

\sanzijing{北元魏、分東西、 \\宇文周、與高齊、} % 原文
{北方北朝的拓拔珪建立北魏後,他注重禮樂及教育,又施
行漢化政策,改姓元故稱元魏,在歷史上頗為有名。至孝
武帝時分裂為東西魏,不久,宇文周篡西魏,建立北周,
高洋篡東魏建立北齊。} % 解释
{} % 解读

\sanzijing{迨至隋、一土宇、 \\不再傳、失統緒、} % 原文
{等到楊堅建立了隋朝,才結束南北朝,統一天下(中國)
,即為隋文帝。文帝深知民間疾苦,一生勤儉愛民,只可
惜識人不明,因改立次子楊廣為太子,種下禍因;楊廣即
隋煬帝,荒淫無道,剛愎自用又好大喜功,連年征討的結
果,天下大亂民不聊生,引起各方聲討,只傳了一代,才
三十八年隋朝就滅亡了。} % 解释
{} % 解读




\sanzijing{唐高祖、起義師、 \\除隋亂、創國基、} % 原文
{唐高祖李淵倡導仁義之師,平定隋朝的亂事,建立了唐朝
二百八十九年的國基。} % 解释
{} % 解读

\sanzijing{二十傳、三百載、 \\梁滅之、國乃改、} % 原文
{唐朝傳了二十代,國運將近三百年,直到朱全忠滅了唐朝
,才把國號改為後梁,為了與前面的朝代有所區別,因此
另加一「後」字。} % 解释
{} % 解读

\sanzijing{梁唐晉、及漢周、 \\稱五代、皆有由、} % 原文
{後梁、後唐、後晉、後漢及後周,這五個朝代稱為五代。
五代的生命都很短暫,其興亡都有原因。} % 解释
{} % 解读

\sanzijing{炎宋興、受周禪、 \\十八傳、南北混、} % 原文
{趙匡胤接受後周恭帝的禪讓,(實際是逼恭帝遜位)建立
宋朝,因為重文輕武只提倡文治不講究軍事,國勢大衰,
北宋和南宋一共傳了十八代,被元朝所統一。} % 解释
{} % 解读

\sanzijing{遼與金、皆稱帝、 \\元滅金、絕宋世、} % 原文
{契丹族所建立的遼國與女真族所建立的金國都曾在中國的
版圖上稱皇帝建國家,後來金國滅了遼國。北方的蒙古人
武力強盛,到元太宗時滅了金國,傳至元世祖忽必烈時終
於滅了南宋,建立元朝。} % 解释
{} % 解读

\sanzijing{輿圖廣、超前代、 \\九十年、國祚廢、} % 原文
{蒙古帝國的版圖之大是前所未有的,比中國最強盛的漢朝和
唐朝還要大。卻因為種族歧視、宗教迫害與高壓政策等因素
(重武力輕文治),只傳了九十年就被朱元璋推翻了。明太
祖興兵起義,南征北討十八年,終於完成統一大業,改國號
為明,訂年號為洪武,在南京建都。} % 解释
{} % 解读






\sanzijing{太祖興、國大明、 \\號洪武、都金陵、} % 原文
{明太祖朱元璋起兵推翻了元朝的統治,建立了明朝
,改年號「洪武」定都金陵。} % 解释
{} % 解读

\sanzijing{迨成祖、遷燕京、 \\十六世、至崇禎、} % 原文
{等到明成祖的時候,將國都遷往燕京,又傳了十六代,到
崇禎皇帝時就結束了。} % 解释
{} % 解读

\sanzijing{權閹肆、寇如林、 \\李闖出、神器焚、} % 原文
{明朝之所以滅亡,是由於太監弄權政治腐敗,加上稅賦重
,人民負擔不起,於是盜匪流寇四起,當時最有名的土匪
流寇李自成,自稱闖王,率兵直闖北京,崇禎眼見大勢已
去,自縊于煤山,結束了明朝二百七十七年的歷史。} % 解释
{} % 解读

\sanzijing{清世祖、膺景命、 \\靖四方、克大定、} % 原文
{清世祖自稱接受天命,入主中原平定各地的流寇作亂,使
天下恢復安定。清世祖是滿清王朝的第一位皇帝。} % 解释
{} % 解读

\sanzijing{由康雍、歷乾嘉、 \\民安富、治績誇、} % 原文
{後來傳位給康熙,雍正歷經乾隆、嘉慶,一百多年間國家富強
,人民生活安定富裕,都是由於政治清明,治理的績效良好所
致,外族統治能夠文武並重,實在值得讚賞。} % 解释
{} % 解读

\sanzijing{道咸間、變亂起、 \\始英法、擾都鄙、} % 原文
{到了道光咸豐年間,國勢逐漸衰弱,內憂外患紛紛而起,內亂
是太平天國的興起,外患則是英法兩國為了通商問題,時常來
騷擾侵犯,咸豐帝死後,慈禧太后掌權,政治倫理蕩然,加上
無能又專斷,寵信宦官且聽信小人讒言,朝政大亂,種下清朝
淪亡的禍因。} % 解释
{(英商在福建、廣東沿海省份販賣鴉片,引誘中
國人吸毒,因為林則徐的禁煙,終於爆發了鴉片戰爭,中國戰
敗,簽訂了喪權辱國的南京條約。)} % 解读





\sanzijing{同光後、宣統弱、 \\傳九帝、滿清歿、} % 原文
{同治光緒皇帝之後,更由於列強入侵,清廷無能,屢戰屢敗
,頻頻割地賠,傳到第九位皇帝宣統,國勢更是積弱不堪,
國父孫中山先生,為救亡圖存起來革命,領導革命先烈拋頭
顱灑熱血,推翻滿清建立民國。} % 解释
{} % 解读

\sanzijing{革命興、廢帝制、 \\立憲法、建民國、} % 原文
{國父孫中山先生領導革命先烈,拋頭顱灑熱血,廢除君主
制度,建立東南亞第一個民主共和國,制訂五權憲法,建
立中華民國。} % 解释
{} % 解读

\sanzijing{古今史、全在茲、 \\載治亂、知興衰、} % 原文
{古往今來所發生的歷史,上自三皇五帝下至清朝共有廿五個
朝代全在這裡。從各朝代的治亂過程中,可以瞭解到興衰的
原因,我們應當記取歷史的教訓,才不會重蹈覆轍。} % 解释
{} % 解读

\sanzijing{史雖繁、讀有次、 \\史記一、漢書二、} % 原文
{史書雖然繁多,研讀時仍需依照次第,才能明白其中的道理。
首先要讀史記其次是漢書、後漢書,三國志第四,這四種書號稱「四史」內容最為精要允當,同時還要尋求其他的經書來證實,並參考資治通鑑以瞭解歷史的全貌,避免以偏蓋全。} % 解释
{} % 解读

\sanzijing{後漢三、國志四、 \\兼證經、參通鑑、} % 原文
{首先要讀史記其次是漢書、後漢書,三國志第四,這四種書號稱「四史」內容最為精要允當,同時還要尋求其他的經書來證實,並參考資治通鑑以瞭解歷史的全貌,避免以偏蓋全。} % 解释
{} % 解读

\sanzijing{讀史者、考實錄、 \\通古今、若親目、} % 原文
{研讀歷史的人,必須要考察記載歷史事實的資料,有一分證據
,說一分話,有十分證據,說十分話。如此才能真正通曉古往
今來的歷史,就如同親眼目睹一樣。} % 解释
{(唐太宗:以銅為鏡可以正衣冠,以人為鏡可以知得失,
以史為鏡可以知興替。)} % 解读






\sanzijing{口而誦、心而惟、 \\朝於斯、夕於斯、} % 原文
{易解:讀書的方法不但要用口去讀去背,還要用心去思考,
不但 白天用功,晚上也要不斷地精進,才能熟記不忘。} % 解释
{(讀書五到:眼到、口到、耳到、心到、手到。)
(博學之、審問之、慎思之、明辨之、篤行之。)} % 解读

\sanzijing{昔仲尼、師項橐、 \\古聖賢、尚勤學、} % 原文
{從前孔老夫子聽說魯國有一位七歲神童項橐,雖然只有七歲
,夫子依然把他當作老師一般請益。像孔老夫子這樣的聖賢
,還這樣不恥下問,我們應該見賢思齊!宋朝趙普好讀論語
,雖然貴為宰相仍然勤奮的學習,並以「半部論語治天下」
,傳為美談。} % 解释
{} % 解读

\sanzijing{趙中令、讀魯論、 \\彼既仕、學且勤、} % 原文
{趙普以半部論語協助宋太祖趙匡胤治天下,又以半部論語
協助宋太宗治天下,輔佐兩朝國君施政皆以論語,可見論
語之殊勝可貴。} % 解释
{} % 解读

\sanzijing{彼蒲編、削竹簡、 \\彼無書、且知勉、} % 原文
{西漢時的溫舒,家貧無力買書,於是利用牧羊時編織蒲草,
將借來的書抄在蒲蓆上,還有一位公孫弘,削竹子做成竹簡
,把借來的書,抄在上面。他們雖然無錢買書,卻能刻苦自
勵,終於成為大學問家,為國家成就一番事業。} % 解释
{} % 解读

\sanzijing{頭懸梁、錐刺股、 \\彼不教、自勤苦、} % 原文
{避免打瞌睡,於是在頭髮上綁了繩子,懸掛在頭頂的木樑
上,當他打瞌睡時,繩子扯動頭髮,就會因此痛醒,再繼
續用功。戰國時的蘇秦,發憤勤學,每當他疲倦昏昏欲睡
的時候,就用錐子刺自己的大腿,讓自己清醒,提醒自己
不能懈怠,他們這樣刻苦用功,都不是別人要求、教導的
,全是自動自發奮發圖強的。} % 解释
{} % 解读

\sanzijing{如囊螢、如映雲、 \\家雖貧、學不輟、} % 原文
{晉朝時的車胤,由於家中貧苦,無錢買油點燈讀書,於是
他就抓些螢火蟲放在網袋中,利用一閃一閃的微弱螢光來
讀書。另外一位名叫孫康,每到夜晚時,就利用雪地上的
反光來讀書,他們家雖窮苦,並沒有停止學習。} % 解释
{} % 解读






\sanzijing{如負薪、如掛角、 \\身雖勞、猶苦卓、} % 原文
{漢人朱買臣,家境貧寒,以砍柴為生,卻常常利用砍柴的
空檔讀書,每次背柴回家的路上,都是一路背誦回家。
隋朝的李密,平日為人放牛,卻仍一心向學,常常把書掛
在牛角上苦讀。他們為了工作謀生雖然身體勞苦,依然堅
苦卓絕的求學,奮發向上的精神,值得我們學習效法。} % 解释
{} % 解读

\sanzijing{蘇老泉、二十七、 \\始發憤、讀書籍、} % 原文
{宋朝的文學家蘇洵,號老泉,二十七歲時忽然覺悟,
開始發憤讀書。} % 解释
{} % 解读

\sanzijing{彼既老、猶悔遲、 \\爾小生、宜早思、} % 原文
{他因為年紀大了,才後悔讀書太晚。你們這些年輕的學子
,應該有所警惕,及早用功讀書。} % 解释
{} % 解读

\sanzijing{若梁灝、八十二、 \\對大廷、魁多士、} % 原文
{五代時,後宋的梁灝,八十二歲還能考中進士,而且在
朝廷的殿試中對答如流,脫穎而出成為狀元。} % 解释
{} % 解读

\sanzijing{彼既成、眾稱異、 \\爾小生、宜立志、} % 原文
{像梁灝年紀雖大仍在用功,大家都稱讚他不平凡,你們
這些年輕學子,應該立定志向及早用功。} % 解释
{} % 解读

\sanzijing{瑩八歲、能詠詩、 \\泌七歲、能賦碁、} % 原文
{北齊的祖瑩,自幼好學,八歲就能吟詩,唐朝的李泌,
七歲時就能以棋為題作賦。} % 解释
{} % 解读






\sanzijing{彼穎悟、人稱奇、 \\爾幼學、當效之、} % 原文
{他們的聰明才智,人人稱奇,你們應當能從小努力,以
他們為模範,好好的學習效法。} % 解释
{} % 解读

\sanzijing{蔡文姬、能辨琴、 \\謝道韞、能詠吟、} % 原文
{東漢末年的蔡文姬,從小便能分辨琴聲的好壞,晉朝宰相
謝安的姪女謝道韞能出口成詩。} % 解释
{} % 解读

\sanzijing{彼女子、且聰敏、 \\爾男子、當自警、} % 原文
{他們這些女孩子,既聰明又敏捷,你們這些男生應當自我警惕
,好好努力,不要分男女,只要是好榜樣就要學習。} % 解释
{} % 解读

\sanzijing{唐劉晏、方七歲、 \\舉神童、作正字、} % 原文
{唐朝的劉晏,七歲時就飽讀詩書,通過童子科的考試,作了
翰林院的正字官,負責校對典籍,刊正文字等工作。} % 解释
{} % 解读

\sanzijing{彼雖幼、身已仕、 \\爾幼學、勉而致、} % 原文
{他雖然年紀幼小,卻已經任職做官,你們從小就要學習,
只要勤勉努力,也是可以做到的。} % 解释
{} % 解读

\sanzijing{有為者、亦若是} % 原文
{一個有志氣的人,肯上進努力,也可以像他們一樣,萬古流芳。} % 解释
{} % 解读






\sanzijing{犬守夜、雞司晨、 \\苟不學、曷為人、} % 原文
{狗會在晚上守夜看門,防止盜賊入侵。公雞會在早晨報曉 ,
提醒人們天亮了;畜生都能忠於職守,我們如果不能善用秉
賦,不知道上進,只是苟且度日,哪還有什麼資格做 人?} % 解释
{} % 解读

\sanzijing{贊吐絲、蜂釀蜜、 \\人不學、不如物、} % 原文
{蠶能吐絲,供給人類作衣服的原料,蜜蜂能採花釀蜜,供人
食用;人如果不肯努力學習,豈不是連這些小昆蟲都不如嗎
?} % 解释
{} % 解读

\sanzijing{幼而學、壯而行、 \\上致君 、下澤民、} % 原文
{一個人在年幼時,就要努力求學,長大後,應該力行實踐
所學到的學問,對上能輔佐長官報效國家,對下能造福社
會人民。} % 解释
{} % 解读

\sanzijing{揚名聲、顯父母、 \\光於前、裕於後、} % 原文
{這樣不但可以得到好名聲,使父母覺得光榮、欣慰,更能
光宗耀祖,對後代子孫是一種啟發,一種典範,一種真正
的庇蔭。} % 解释
{} % 解读

\sanzijing{人遺子、金滿籯、 \\我教子、惟一經、} % 原文
{一般人留給子孫的是滿箱的財寶,我卻不同,只有一部三
字經,用來教導子孫好好讀書,明白做人處事的道理。} % 解释
{} % 解读

\sanzijing{勤有功、戲無益、 \\戒之哉、宜勉力、} % 原文
{只要肯勤勞努力的學習,就會有成果,如果只是嬉戲和遊玩
不肯上進,對自己對父母師長,都是沒有益處,要時常警惕
,好好努力。} % 解释
{} % 解读

\end{document}
