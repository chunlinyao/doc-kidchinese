%# -*- coding: utf-8 -*-
%!TEX encoding = UTF-8 Unicode
%!TEX TS-program = xelatex
% vim:ts=4:sw=4
%
% 以上设定默认使用 XeLaTex 编译,并指定 Unicode 编码,供 TeXShop 自动识别
\documentclass[avery5371,grid]{flashcards}


\newcommand{\doctitle}{小學生必背古詩百首}
\newcommand{\docauthor}{} %{Yunhui Fu}
\newcommand{\dockeywords}{{中文}{古詩}}
\newcommand{\docsubject}{}


\cardfrontstyle[\large\slshape]{headings}
\cardbackstyle{empty}
%\cardfrontfoot{中文}
\cardfrontfoot{}%{\doctitle}


%# -*- coding: utf-8 -*-
%!TEX encoding = UTF-8 Unicode
%!TEX TS-program = xelatex
% vim:ts=4:sw=4
%
% 以上设定默认使用 XeLaTex 编译,并指定 Unicode 编码,供 TeXShop 自动识别


\newcommand\mymainfont{Noto Serif}%{Times New Roman} %{DejaVu Serif}
\newcommand\mymonofont{DejaVu Sans Mono}%{FreeMono} %{WenQuanYi Micro Hei Mono} %{Monaco}
\newcommand\myboldfont{DejaVu Sans Mono} %{WenQuanYi Micro Hei Mono}%{AR PL UKai CN}%{YaHei Consolas Hybrid}%{黑体}%{標楷體}
\newcommand\mysansfont{DejaVu Sans}%{FreeSans}
\newcommand\myitalicfont{DejaVu Serif}%{Times New Roman} %{Garamond}

\newcommand\mycjkboldfont{Microsoft YaHei} %{WenQuanYi Micro Hei Mono}%{Adobe Heiti Std}%{AR PL UKai CN}%{YaHei Consolas Hybrid}%{黑体}%{標楷體}
\newcommand\mycjkitalicfont{全字庫正楷體} %{Adobe Kaiti Std}
\newcommand\mycjkmainfont{WenyueType GutiFangsong (Noncommercial Use)}%{全字庫正楷體} %{Adobe Song Std} %{AR PL UMing CN}%{仿宋}%{宋体}%{新宋体}%{文鼎PL新宋}%
\newcommand\mycjksansfont{MingLiU} %{Adobe Ming Std}
\newcommand\mycjkmonofont{DejaVu Sans YuanTi Mono}%{WenQuanYi Micro Hei Mono}%{AR PL UMing CN}%{WenQuanYi Micro Hei Mono}


%\usepackage[nofonts]{ctex} %adobefonts
\usepackage[adobefonts]{ctex} %adobefonts
%\usepackage[fallback]{xeCJK}


\newCJKfontfamily{\mykaiti}{全字庫正楷體}%{AR PL UKai TW}%{全字庫正楷體}
\newCJKfontfamily{\myfangsong}{WenyueType GutiFangsong (Noncommercial Use)}



\usepackage{ifthen}
\usepackage{ifpdf}
\usepackage{ifxetex}
\usepackage{ifluatex}


\usepackage{color}
\usepackage[rgb]{xcolor}



\xeCJKsetup{AutoFallBack = true} % you have to use this, since fallback won't work as the xeCJK option after the ctex

\PassOptionsToPackage{
    BoldFont,  % 允許粗體
    SlantFont, % 允許斜體
    CJKnumber,
    CJKtextspaces,
    }{xeCJK}
\defaultfontfeatures{Mapping=tex-text} % 如果沒有它,會有一些 tex 特殊字符無法正常使用,比如連字符。

\xeCJKsetup {
    CheckSingle = true,
    AutoFakeBold = false,
AutoFakeSlant = false,
    CJKecglue = {},
    PunctStyle = kaiming,
KaiMingPunct+ = {:;},
}

\PassOptionsToPackage{CJKchecksingle}{xeCJK}
%\defaultCJKfontfeatures{Scale=0.5}
%\LoadClass[c5size,openany,nofonts]{ctexbook}
\XeTeXlinebreaklocale "zh"                      % 重要,使得中文可以正確斷行!
\XeTeXlinebreakskip = 0pt plus 1pt minus 0.1pt  % 给予TeX断行一定自由度
\linespread{1.3}                                % 1.3倍行距

\setCJKmainfont[BoldFont=\mycjkboldfont, ItalicFont=\mycjkitalicfont]{\mycjkmainfont}
\setCJKsansfont{\mycjksansfont}%{Adobe Ming Std} %{AR PL UMing CN} %{Microsoft YaHei}
\setCJKmonofont{\mycjkmonofont}



\ifxetex % xelatex
\else
    %The cmap package is intended to make the PDF files generated by pdflatex "searchable and copyable" in acrobat reader and other compliant PDF viewers.
    \usepackage{cmap}%
\fi
% ============================================
% Check for PDFLaTeX/LaTeX
% ============================================
\newcommand{\outengine}{xetex}
\newif\ifpdf
\ifx\pdfoutput\undefined
  \pdffalse % we are not running PDFLaTeX
  \ifxetex
    \renewcommand{\outengine}{xetex}
  \else
    \renewcommand{\outengine}{dvipdfmx}
  \fi
\else
  \pdfoutput=1 % we are running PDFLaTeX
  \pdftrue
  \usepackage{thumbpdf}
  \renewcommand{\outengine}{pdftex}
  \pdfcompresslevel=9
\fi
\usepackage[\outengine,
    bookmarksnumbered, %dvipdfmx
    %% unicode, %% 不能有unicode选项,否则bookmark会是乱码
    colorlinks=true,
    citecolor=red,
    urlcolor=blue,        % \href{...}{...} external (URL)
    filecolor=red,      % \href{...} local file
    linkcolor=black, % \ref{...} and \pageref{...}
    breaklinks,
    pdftitle={\doctitle},
    pdfauthor={\docauthor},
    pdfsubject={\docsubject},
    pdfkeywords={\dockeywords},
    pdfproducer={Latex with hyperref},
    pdfcreator={pdflatex},
    %%pdfadjustspacing=1,
    pdfborder=1,
    pdfpagemode=UseNone,
    pagebackref,
    bookmarksopen=true]{hyperref}

% --------------------------------------------
% Load graphicx package with pdf if needed 
% --------------------------------------------
\ifxetex    % xelatex
    \usepackage{graphicx}
\else
    \ifpdf
        \usepackage[pdftex]{graphicx}
        \pdfcompresslevel=9
    \else
        \usepackage{graphicx} % \usepackage[dvipdfm]{graphicx}
    \fi
\fi

%\newCJKfontfamily{\yanti}{\mycjkmainfont}
%\newenvironment{shicibody}{%
%\begin{verse}\centering\yanti\large\hspace{13pt}}{\end{verse}}


\usepackage{xstring}

%The baseline-skip should be set to roughly 1.2x the font size.
%\fontsize{size}{baselineskip}
%\fontsize{50}{60}

% 计算所能容纳的字数
\newcommand\shizishow[1]{
\StrLen{#1}[\mystringlen]
%\def\mythresh{1}
    \ifthenelse{\mystringlen < 2}{
        {\fontsize{130}{156} #1}
    }{ \ifthenelse{\mystringlen < 3}{
        {\fontsize{110}{122} #1}%{\fontsize{100}{120} #1}
    }{ \ifthenelse{\mystringlen < 4}{
        {\fontsize{78}{94} #1}%{\fontsize{90}{108} #1}
    }{ \ifthenelse{\mystringlen < 5}{
        {\fontsize{55}{66} #1}
    }{ \ifthenelse{\mystringlen < 7}{
        {\fontsize{70}{84} #1}
    }{ \ifthenelse{\mystringlen < 9}{
        {\fontsize{55}{66} #1}
    }{ \ifthenelse{\mystringlen < 16}{
        {\zihao{0} #1} %5x4=20
    }{ \ifthenelse{\mystringlen < 49}{
        {\zihao{1} #1} % 8x6=48
    }{ \ifthenelse{\mystringlen < 55}{
        {\Huge #1} % 9x6=54
    }{ \ifthenelse{\mystringlen < 61}{
        {\zihao{2} #1} % 10x6=60
    }{ \ifthenelse{\mystringlen < 78}{
        {\huge #1} % 11x7=77
    }{ \ifthenelse{\mystringlen < 105}{
        {\LARGE #1} % 13x8=104
    }{ \ifthenelse{\mystringlen < 113}{
        {\zihao{3} #1} % 14x8=112
    }{ \ifthenelse{\mystringlen < 129}{
        {\Large #1} % 16x8=128 %{\zihao{4} #1} % 16x8=128
    }{ \ifthenelse{\mystringlen < 153}{
        {\large #1} % 19x8=152, or 19x9=171
    }{ \ifthenelse{\mystringlen < 198}{
        {\zihao{5} #1} % 22x9=198
    }{ \ifthenelse{\mystringlen < 207}{
        {\normalsize #1} % 23x9=207
    }{ \ifthenelse{\mystringlen < 234}{
        {\small #1} % 26x9=234
    }{ \ifthenelse{\mystringlen < 261}{
        {\footnotesize #1} % 29x9=261
    }{ \ifthenelse{\mystringlen < 279}{
        {\zihao{6} #1} % 31x9=279
    }{ \ifthenelse{\mystringlen < 297}{
        {\scriptsize #1} % 33x9=297
    }{ \ifthenelse{\mystringlen < 378}{
        {\zihao{7} #1} % 42x9=378
    }{
        {\tiny #1} % 46x9=414 %{\zihao{8} #1} % 46x9=414
    }}}}}}}}}}}}}}}}}}}}}}
}

% 计算包含拼音时所能容纳的字数
\newcommand\shizipy[1]{
\StrLen{#1}[\mystringlen]
\xpinyin*{
    \ifthenelse{\mystringlen < 6}{
        {\zihao{0} #1} %5x4=20
    }{ \ifthenelse{\mystringlen < 18}{
        {\zihao{1} #1} % 8x6=48
    }{ \ifthenelse{\mystringlen < 22}{
        {\Huge #1} % 9x6=54
    }{ \ifthenelse{\mystringlen < 31}{
        {\zihao{2} #1} % 10x6=60
    }{ \ifthenelse{\mystringlen < 34}{
        {\huge #1} % 11x7=77
    }{ \ifthenelse{\mystringlen < 53}{
        {\LARGE #1} % 13x8=104
    }{ \ifthenelse{\mystringlen < 57}{
        {\zihao{3} #1} % 14x8=112
    }{ \ifthenelse{\mystringlen < 65}{
        {\Large #1} % 16x8=128 %{\zihao{4} #1} % 16x8=128
    }{ \ifthenelse{\mystringlen < 77}{
        {\large #1} % 19x8=152, or 19x9=171
    }{ \ifthenelse{\mystringlen < 89}{
        {\zihao{5} #1} % 22x9=198
    }{ \ifthenelse{\mystringlen < 93}{
        {\normalsize #1} % 23x9=207
    }{ \ifthenelse{\mystringlen < 105}{
        {\small #1} % 26x9=234
    }{ \ifthenelse{\mystringlen < 117}{
        {\footnotesize #1} % 29x9=261
    }{ \ifthenelse{\mystringlen < 125}{
        {\zihao{6} #1} % 31x9=279
    }{ \ifthenelse{\mystringlen < 133}{
        {\scriptsize #1} % 33x9=297
    }{ \ifthenelse{\mystringlen < 169}{
        {\zihao{7} #1} % 42x9=378
    }{
        {\tiny #1} % 46x9=414 %{\zihao{8} #1} % 46x9=414
    }}}}}}}}}}}}}}}}
}
}


% 计算在comment中(三字经)能容纳的字数
\newcommand\sanzicomments[1]{
\StrLen{#1}[\mystringlen]
%\def\mythresh{1}
    \ifthenelse{\mystringlen < 7}{
        {\zihao{0} #1} %5x4=20
    }{ \ifthenelse{\mystringlen < 22}{
        {\zihao{1} #1} % 8x6=48
    }{ \ifthenelse{\mystringlen < 25}{
        {\Huge #1} % 9x6=54
    }{ \ifthenelse{\mystringlen < 38}{
        {\zihao{2} #1} % 10x6=60
    }{ \ifthenelse{\mystringlen < 42}{ %
        {\huge #1} % 11x7=77
    }{ \ifthenelse{\mystringlen < 63}{
        {\LARGE #1} % 13x8=104
    }{ \ifthenelse{\mystringlen < 64}{
        {\zihao{3} #1} % 14x8=112
    }{ \ifthenelse{\mystringlen < 87}{
        {\Large #1} % 16x8=128 %{\zihao{4} #1} % 16x8=128
    }{ \ifthenelse{\mystringlen < 125}{
        {\large #1} % 19x8=152, or 19x9=171
    }{ \ifthenelse{\mystringlen < 173}{
        {\zihao{5} #1} % 22x9=198
    }{ \ifthenelse{\mystringlen < 205}{
        {\normalsize #1} % 23x9=207
    }{ \ifthenelse{\mystringlen < 232}{
        {\small #1} % 26x9=234
    }{ \ifthenelse{\mystringlen < 259}{
        {\footnotesize #1} % 29x9=261
    }{ \ifthenelse{\mystringlen < 277}{
        {\zihao{6} #1} % 31x9=279
    }{ \ifthenelse{\mystringlen < 295}{
        {\scriptsize #1} % 33x9=297
    }{ \ifthenelse{\mystringlen < 376}{
        {\zihao{7} #1} % 42x9=378
    }{
        {\tiny #1} % 46x9=414 %{\zihao{8} #1} % 46x9=414
    }}}}}}}}}}}}}}}}
}



% 诗词
\newcommand\shici[5]{
\begin{flashcard}[#1 -- #2 (#3) #4]{%
{\mykaiti \fontsize{20}{24} {#5}} %\\
}
\vspace*{\stretch{1}}
\begin{center}

{\mykaiti \zihao{1} {#2}}\\ \vspace*{\stretch{.5}}
{\large(#3)}\\ \vspace*{\stretch{.25}}
{\LARGE #4}

\end{center}
\vspace*{\stretch{1}}
\end{flashcard}
}



% 识字

%\usepackage[overlap,CJK]{ruby}
\usepackage{xpinyin}

% 如果字数多,则略写
\newcommand\shizititle[2]{
  \StrLen{(#1) #2}[\mystringlen]%
  \ifthenelse{\mystringlen > 24}{
    \StrLeft{(#1) #2}{20}......\StrRight{(#1) #2}{4}
  }{
  (#1) #2
  }
}
\newcommand\shizi[3]{
\begin{flashcard}[\shizititle{#1}{#2}]{ %
{\mykaiti \shizipy{#2}}

{\myfangsong \normalsize #3} %
} %
\vspace*{\stretch{1}}
\begin{center}
%ēéěè
%\ruby{莉}{li}
%\xpinyin*{床前明月光,疑是地上霜。举头望明月,低头思故乡。}
%\begin{pinyinscope}
%床前明月光,疑是地上霜。举头望明月,低头思故乡。
%\end{pinyinscope}

{\mykaiti \shizishow{#2}}

\end{center}
\vspace*{\stretch{1}}
\end{flashcard}
}


\newcounter{fcsanzi}\setcounter{fcsanzi}{0}

% 三字经
\newcommand\sanzijing[3]{
\stepcounter{fcsanzi}
\begin{flashcard}[]{
\mykaiti \fontsize{26}{\baselineskip}\selectfont \xpinyin*{#1}
}
\vspace*{\stretch{1}}
%\begin{center}
%\normalsize
{
\myfangsong
\sanzicomments{
【解释】 #2

〖解读〗 #3
}
}
%\end{center}
\vspace*{\stretch{1}}
\end{flashcard}
}





\newcommand\docshowcopyright{
\begin{flashcard}[Copyright \& License]{Copyright \copyright \, 2014 Yunhui Fu\\
Some rights reserved.}
\vspace*{\stretch{1}}
These flashcards and the accompanying \LaTeX \, source code are licensed
under a Creative Commons Attribution--NonCommercial--ShareAlike 2.5 License.  
For more information, see creativecommons.org.  You can contact the author at:
\begin{center}
\begin{small}\tt yhfudev at gmail com\end{small}

\medskip
File last updated on \today, \\
at \currenttime
\end{center}
\vspace*{\stretch{1}}
\end{flashcard}

\begin{flashcard}[版权申明]{版权所有 \copyright \, 2014 Yunhui Fu\\
有些版权保留.}
\vspace*{\stretch{1}}
闪卡及其 \LaTeX \, 源代码在
署名-相同方式共享 2.5 下保护.
参见 creativecommons.org.  你可以联系作者:
\begin{center}
\begin{small}\tt yhfudev at gmail com\end{small}

\medskip
文件最近更新: \today, \\
\currenttime
\end{center}
\vspace*{\stretch{1}}
\end{flashcard}
}






\newcommand\docshowtitle[3]{
\begin{flashcard}[]{ %
{\mykaiti \Huge{#1}}

{\myfangsong \normalsize #2} %
} %
\vspace*{\stretch{1}}
\begin{center}

{\mykaiti \shizishow{#3}}

\end{center}
\vspace*{\stretch{1}}
\end{flashcard}
}







\usepackage{datetime}

\begin{document}

\docshowcopyright
\docshowtitle{\doctitle}{\docauthor}{%
使用双面打印,然后按线剪下。
}


\shici{古詩}{山村咏怀}{宋}{邵康节}{ %
一去二三里,\\
煙村四五家。\\
亭台六七座,\\
八九十枝花。\\
}

\shici{古詩}{詠鵝}{唐}{駱賓王}{ %
鵝鵝鵝,\\
曲項向天歌。\\
白毛浮綠水,\\
紅掌撥清波。\\
}

\shici{古詩}{偈颂}{南宋}{道川}{
遠觀山有色。近聽水無聲。\\
春去花猶在。人來鳥不驚。\\
頭頭皆顯露。物物體元平。\\
如何言不會。只為太分明。\\
}

\shici{古詩}{畫雞}{明}{唐寅}{
頭上紅冠不用裁,\\
滿身雪白走將來。\\
平生不敢輕言語,\\
一叫千門萬戶開。\\
}

\shici{古詩}{靜夜思}{唐}{李白}{
床前明月光,\\
疑是地上霜。\\
擧頭望明月,\\
低頭思故鄉。\\
}

\shici{古詩}{春曉}{唐}{孟浩然}{
春眠不覺曉,\\
處處聞啼鳥。\\
夜來風雨聲,\\
花落知多少。\\
}

\shici{古詩}{村居}{清}{高鼎}{
草長鶯飛二月天,\\
拂堤楊柳醉春煙。\\
兒童散學歸來早,\\
忙趁東風放紙鳶。\\
}

\shici{古詩}{所見}{清}{袁枚}{
牧童騎黄牛,\\
歌聲振林樾。\\
意欲捕鳴蟬,\\
忽然閉口立。\\
}

\shici{古詩}{小池}{宋}{楊萬里}{
泉眼無聲惜細流,\\
樹陰照水愛晴柔。\\
小荷才露尖尖角,\\
早有蜻蜓立上頭。\\
}

\shici{古詩}{贈劉景文}{宋}{蘇軾}{
荷盡已無擎雨蓋, \\
菊殘猶有傲霜枝。\\
一年好景君須記,\\
正是橙黄橘綠時。\\
}

\shici{古詩}{山行}{唐}{杜牧}{
遠上寒山石徑斜,\\
白雲深處有人家。\\
停車坐愛楓林晚,\\
霜葉紅於二月花。\\
}

\shici{古詩}{回鄉偶書}{唐}{賀知章}{
少小離家老大回,\\
鄉音無改鬢毛衰。\\
兒童相見不相識,\\
笑問客從何處來。\\
}

\shici{古詩}{贈汪倫}{唐}{李白}{
李白乘舟將欲行,\\
忽聞岸上踏歌聲。\\
桃花潭水深千尺,\\
不及汪倫送我情。\\
}

\shici{古詩}{賦得古原草送別}{唐}{白居易}{
離離原上草,
一歲一枯榮。\\
野火燒不盡,
春風吹又生。\\
遠芳侵古道,晴翠接荒城。\\
又送王孫去,萋萋滿別情。\\
}

\shici{古詩}{宿新市徐公店}{宋}{楊萬里}{
籬落疏疏一徑深,\\
樹頭花落未成陰。\\
兒童急走追黄蝶,\\
飛入菜花無處尋。\\
}

\shici{古詩}{敕勒歌}{北朝}{樂府}{
敕勒川,陰山下,\\
天似穹廬,
籠蓋四野。\\
天蒼蒼,野茫茫,\\
風吹草低見牛羊。\\
}

\shici{古詩}{望廬山瀑布}{唐}{李白}{
日照香鑪生紫煙,\\
遙看瀑布掛前川。\\
飛流直下三千尺,\\
疑是銀河落九天。\\
}

\shici{古詩}{絕句}{唐}{杜甫}{
兩個黄鸝鳴翠柳,\\
一行白鷺上青天。\\
窗含西嶺千秋雪,\\
門泊東吳萬里船。\\
}

\shici{古詩}{江畔獨步尋花}{唐}{杜甫}{
黄師塔前江水東,\\
春光懶困倚微風。\\
桃花一簇開無主,\\
可愛深紅愛淺紅?\\
}

\shici{古詩}{詠柳}{唐}{賀知章}{
碧玉妝成一樹高,\\
萬條垂下綠絲絛。\\
不知細葉誰裁出,\\
二月春風似剪刀。\\
}

\shici{古詩}{江雪}{唐}{柳宗元}{
千山鳥飛絕,\\
萬徑人蹤滅。\\
孤舟蓑笠翁,\\
獨釣寒江雪。\\
}

\shici{古詩}{春日}{宋}{朱熹}{
勝日尋芳泗水濱,\\
無邊光景一時新。\\
等閑識得東風面,\\
萬紫千紅總是春。\\
}

\shici{古詩}{池上}{唐}{白居易}{
小娃撑小艇,\\
偷采白蓮回。\\
不解藏蹤蹟,\\
浮萍一道開。\\
}

\shici{古詩}{清明}{唐}{杜牧}{
清明時節雨紛紛,\\
路上行人欲斷魂。\\
借問酒家何處有,\\
牧童遙指杏花村。\\
}

\shici{古詩}{尋隱者不遇}{唐}{賈島}{
松下問童子,\\
言師采藥去。\\
隻在此山中,\\
雲深不知處。\\
}

\shici{古詩}{題臨安邸}{宋}{林升}{
山外青山樓外樓,\\
西湖歌舞幾時休。\\
暖風熏得游人醉,\\
直把杭州作汴州。\\
}

\shici{古詩}{江南}{漢}{樂府民歌}{
江南可采蓮,
蓮葉何田田,\\
魚戲蓮葉間。\\
魚戲蓮葉東,
魚戲蓮葉西,\\
魚戲蓮葉南,
魚戲蓮葉北。\\
}

\shici{古詩}{賦得古原草送别}{唐}{白居易}{
離離原上草,
一歲一枯榮。\\
野火燒不盡,
春風吹又生。\\
遠芳侵古道,
晴翠接荒城。\\
又送王孫去,
萋萋滿别情。\\
}

\shici{古詩}{涼州詞}{唐}{王之渙}{
黄河遠上白雲間,\\
一片孤城萬仞山。\\
羌笛何須怨楊柳,\\
春風不度玉門關。\\
}

\shici{古詩}{樂游原}{唐}{李商隱}{
向晚意不適,\\
驅車登古原。\\
夕陽無限好,\\
隻是近黄昏。\\
}

\shici{古詩}{夏日絕句}{宋}{李清照}{
生當作人傑,\\
死亦爲鬼雄。\\
至今思項羽,\\
不肯過江東。\\
}

\shici{古詩}{憫農(之一)}{唐}{李紳}{
鋤禾日當午,\\
汗滴禾下土。\\
誰知盤中餐,\\
粒粒皆辛苦。\\
}

\shici{古詩}{憫農(之二)}{唐}{李紳}{
春種一粒粟,\\
秋收萬顆子。\\
四海無閑田,\\
農夫猶餓死。\\
}

\shici{古詩}{絕句}{唐}{杜甫}{
遲日江山麗,\\
春風花草香。\\
泥融飛燕子,\\
沙暖睡鴛鴦。\\
}

\shici{古詩}{古朗月行}{唐}{李白}{
小時不識月,\\
呼作白玉盤。\\
又疑瑤台鏡,\\
飛在青雲端。\\
}

\shici{古詩}{塞下曲(之一)}{唐}{盧綸}{
鷲翎金仆姑,\\
燕尾繡蝥弧。\\
獨立揚新令,\\
千營共一呼。
}

\shici{古詩}{塞下曲(之二)}{唐}{盧綸}{
林暗草驚風,\\
將軍夜引弓。\\
平明尋白羽,\\
沒在石棱中。
}

\shici{古詩}{塞下曲(之三)}{唐}{盧綸}{
月黑雁飛高,\\
單于夜遁逃。\\
欲將輕騎逐,\\
大雪滿弓刀。\\
}

\shici{古詩}{塞下曲(之四)}{唐}{盧綸}{
野幕蔽瓊筵,\\
羌戎賀勞旋。\\
醉和金甲舞,\\
雷鼓動山川。
}

\shici{古詩}{登鸛雀樓}{唐}{王之渙}{
白日依山盡,\\
黄河入海流。\\
欲窮千里目,\\
更上一層樓。\\
}

\shici{古詩}{秋夕}{唐}{杜牧}{
銀燭秋光冷畫屏,\\
輕羅小扇撲流螢。\\
天階夜色涼如水,\\
坐看牽牛織女星。\\
}

\shici{古詩}{鄉村四月}{宋}{翁卷}{
綠遍山原白滿川,\\
子規聲里雨如煙。\\
鄉村四月閑人少,\\
才了蠶桑又插田。\\
}

\shici{古詩}{九月九日憶山東兄弟}{唐}{王維}{
獨在異鄉爲異客,\\
每逢佳節倍思親。\\
遙知兄弟登高處,\\
遍插茱萸少一人。\\
}

\shici{古詩}{登飛來峰}{宋}{王安石}{
飛來山上千尋塔,\\
聞說雞鳴見日升。\\
不畏浮雲遮望眼,\\
自緣身在最高層。\\
}

\shici{古詩}{鹿柴}{唐}{王維}{
空山不見人,\\
但聞人語響。\\
返景入深林,\\
複照青苔上。\\
}

\shici{古詩}{江上漁者}{宋}{範仲淹}{
江上往來人,\\
但愛鱸魚美。\\
君看一葉舟,\\
出沒風波里。\\
}

\shici{古詩}{逢雪宿芙蓉山主人}{唐}{劉長卿}{
日暮蒼山遠,\\
天寒白屋貧。\\
柴門聞犬吠,\\
風雪夜歸人。\\
}

\shici{古詩}{元日}{宋}{王安石}{
爆竹聲中一歲除,\\
春風送暖入屠蘇;\\
千門萬戶瞳瞳日,\\
總把新桃換舊符。\\
}

\shici{古詩}{夜書所見}{宋}{葉紹翁}{
蕭蕭梧葉送寒聲,\\
江上秋風動客情。\\
知有兒童挑促織,\\
夜深籬落一燈明。\\
}

\shici{古詩}{望天門山}{唐}{李白}{
天門中斷楚江開,\\
碧水東流至此回。\\
兩岸青山相對出,\\
孤帆一片日邊來。\\
}

\shici{古詩}{飲湖上初晴後雨}{宋}{蘇軾}{
水光瀲灩晴方好,\\
山色空蒙雨亦奇。\\
欲把西湖比西子,\\
淡妝濃抹總相宜。\\
}

\shici{古詩}{小兒垂釣}{唐}{胡令能}{
蓬頭稚子學垂綸,\\
側坐莓苔草映身。\\
路人借問遙招手,\\
怕得魚驚不應人。\\
}

\shici{古詩}{獨坐敬亭山}{唐}{李白}{
眾鳥高飛盡,\\
孤雲獨去閑。\\
相看兩不厭,\\
隻有敬亭山。\\
}

\shici{古詩}{宿建德江}{唐}{孟浩然}{
移舟泊煙渚,\\
日暮客愁新。\\
野曠天低樹,\\
江清月近人。\\
}

\shici{古詩}{舟夜書所見}{清}{查慎行}{
月黑見漁燈,\\
孤光一點螢。\\
微微風簇浪,\\
散作滿河星。\\
}

\shici{古詩}{送元二使安西}{唐}{王維}{
渭城朝雨邑輕塵,\\
客舍青青柳色新。\\
勸君更進一杯酒,\\
西出陽關無故人。\\
}

\shici{古詩}{早發白帝城}{唐}{李白}{
朝辭白帝彩雲間,\\
千里江陵一日還。\\
兩岸猿聲啼不住,\\
輕舟已過萬重山。\\
}

\shici{古詩}{滁洲西澗}{唐}{韋應物}{
獨憐幽草澗邊生,\\
上有黄鸝深樹鳴。\\
春潮帶雨晚來急,\\
野渡無人舟自横。\\
}

\shici{古詩}{菊花}{唐}{元稹}{
秋叢繞舍似陶家,\\
遍繞籬邊日漸斜。\\
不是花中偏愛菊,\\
此花開盡更無花。\\
}

\shici{古詩}{三衢道中}{宋}{曾幾}{
梅子黄時日日晴,\\
小溪泛盡卻山行。\\
綠陰不減來時路,\\
添得黄鸝四五聲。\\
}

\shici{古詩}{清平樂·村居}{宋}{辛棄疾}{
茅檐低小,
溪上青青草。\\
醉里吳音相媚好,
白發誰家翁媼。\\
大兒鋤豆溪東,
中兒正織雞籠,\\
最喜小兒無賴,
溪頭臥剝蓮蓬。\\
}

\shici{古詩}{惠崇春江晚景}{宋}{蘇軾}{
竹外桃花三兩枝,\\
春江水暖鴨先知。\\
蔞蒿滿地蘆芽短,\\
正是河豚欲上時。\\
}

\shici{古詩}{江南春}{唐}{杜牧}{
千里鶯啼綠映紅,\\
水村山郭酒旗風。\\
南朝四百八十寺,\\
多少樓台煙雨中。\\
}

\shici{古詩}{四時田園雜興}{宋}{範成大}{
梅子金黄杏子肥,\\
麥花雪白菜花稀。\\
日長籬落無人過,\\
惟有蜻蜓蛺蝶飛。\\
}

\shici{古詩}{如夢令}{宋}{李清照}{
常記溪亭日暮,
沉醉不知歸路。\\
興盡晚回舟,
誤入藕花深處。\\
爭渡,爭渡,
驚起一灘鷗鷺。\\
}

\shici{古詩}{四時田園雜興}{宋}{範成大}{
晝出耘田夜績麻,\\
村莊兒女各當家。\\
童孫未解供耕織,\\
也傍桑陰學種瓜。\\
}

\shici{古詩}{黄鶴樓送孟浩然之廣陵}{唐}{李白}{
故人西辭黄鶴樓,\\
煙花三月下颺州。\\
孤帆遠影碧空盡,\\
唯見長江天際流。\\
}

\shici{古詩}{題西林壁}{宋}{蘇軾}{
横看成嶺側成峰,\\
遠近高低各不同。\\
不識廬山真面目,\\
隻緣身在此山中。\\
}

\shici{古詩}{楓橋夜泊}{唐}{張繼}{
月落烏啼霜滿天,\\
江楓漁火對愁眠。\\
姑蘇城外寒山寺,\\
夜半鍾聲到客船。\\
}

\shici{古詩}{别董大}{唐}{高適}{
千里黄雲白日曛,\\
北風吹雁雪紛紛。\\
莫愁前路無知己,\\
天下誰人不識君?\\
}

\shici{古詩}{暮江吟}{唐}{白居易}{
一道殘陽鋪水中,\\
半江瑟瑟半江紅。\\
可憐九月初三夜,\\
露似真珠月似弓。\\
}

\shici{古詩}{終南望餘雪}{唐}{祖詠}{
終南陰嶺秀,\\
積雪浮雲端。\\
林表明霽色,\\
城中增暮寒。\\
}

\shici{古詩}{憶江南}{唐}{白居易}{
江南好,
風景舊曾諳。\\
日出江花紅勝火,\\
春來江水綠如藍。\\
能不憶江南。\\
}

\shici{古詩}{漁歌子}{唐}{張志和}{
西塞山前白鷺飛,\\
桃花流水鱖魚肥。\\
青箬笠,綠蓑衣,\\
斜風細雨不須歸。\\
}

\shici{古詩}{游園不值}{宋}{葉紹翁}{
應憐屐齒印蒼苔,\\
小扣柴扉久不開。\\
春色滿園關不住,\\
一枝紅杏出牆來。\\
}

\shici{古詩}{曉出淨慈寺送林子方}{宋}{楊萬里}{
畢竟西湖六月中,\\
風光不與四時同。\\
接天蓮葉無窮碧,\\
映日荷花别樣紅。\\
}

\shici{古詩}{長相思}{清}{納蘭性德}{
山一程,水一程,
身向逾關那畔行,\\
夜深千帳燈。\\
風一更,雪一更,
聒碎鄉心夢不成,\\
故園無此聲。\\

}

\shici{古詩}{西江月·夜行黄沙道中}{宋}{辛棄疾}{
明月别枝驚鵲,
清風半夜鳴蟬。\\
稻花香里說豐年,
聽取蛙聲一片。\\
七八個星天外,
兩三點雨山前。\\
舊時茅店社林邊,
路轉溪橋忽見。\\
}

\shici{古詩}{墨梅}{元}{王冕}{
我家洗硯池頭樹,\\
朵朵花開淡墨痕。\\
不要人誇顏色好,\\
隻留清氣滿乾坤。\\
}

\shici{古詩}{竹石}{清}{鄭燮}{
咬定青山不放松,\\
立根原在破岩中。\\
千磨萬擊還堅勁,\\
任爾東西南北風。\\
}

\shici{古詩}{石灰吟}{明}{於謙}{
千鎚萬鑿出深山,\\
烈火焚燒若等閑。\\
粉身碎骨渾不怕,\\
要留清白在人間。\\
}

\shici{古詩}{泊船瓜洲}{宋}{王安石}{
京口瓜洲一水間,\\
鍾山隻隔數重山。\\
春風又綠江南岸,\\
明月何時照我還?\\
}

\shici{古詩}{游子吟}{唐}{孟郊}{
慈母手中線,
游子身上衣。\\
臨行密密縫,
意恐遲遲歸。\\
誰言寸草心,
報得三春暉。\\
}

\shici{古詩}{菩薩蠻·書江西造口壁}{宋}{辛棄疾}{
鬱孤台下清江水,
中間多少行人淚。\\
西北望長安,
可憐無數山。\\
青山遮不住,
畢竟東流去。\\
江晚正愁餘,
山深聞鷓鴣。\\
}

\shici{古詩}{望洞庭}{唐}{劉禹錫}{
湖光秋月兩相和,\\
潭面無風鏡未磨。\\
遙望洞庭山水色,\\
白銀盤里一青螺。\\
}

\shici{古詩}{蔔算子·詠梅}{宋}{陸游}{
驛外斷橋邊,
寂寞開無主。\\
已是黄昏獨自愁,
更著風和雨。\\
無意苦爭春,
一任群芳妒。\\
零落成泥碾作塵,
隻有香如故。\\
}

\shici{古詩}{長歌行}{漢}{樂府}{
青青園中葵,
朝露待日晞。\\
陽春布德澤,
萬物生光輝。\\
常恐秋節至,
焜黄華葉衰。\\
}

\shici{古詩}{七步詩}{東漢}{曹植}{
煮豆持作羹,
漉豉以爲汁。\\
萁在釜下燃,
豆在釜中泣。\\
本自同根生,
相煎何太急?\\
}

\shici{古詩}{出塞}{唐}{王昌齡}{
秦時明月漢時關,\\
萬里長征人未還。\\
但使龍城飛將在,\\
不教胡馬度陰山。\\
}

\shici{古詩}{春夜喜雨}{唐}{杜甫}{
好雨知時節,
當春乃發生。\\
隨風潛入夜,
潤物細無聲。\\
野徑雲俱黑,
江船火獨明。\\
曉看紅濕處,
花重錦官城。\\
}

\shici{古詩}{示兒}{宋}{陸游}{
死去元知萬事空,\\
但悲不見九州同。\\
王師北定中原日,\\
家祭無忘告乃翁。\\
}

\shici{古詩}{聞官軍收河南河北}{唐}{杜甫}{
劍外忽傳收薊北,
初聞涕淚滿衣裳。\\
卻看妻子愁何在,
漫卷詩書喜欲狂。\\
白日放歌須縱酒,
青春作伴好還鄉。\\
即從巴峽穿巫峽,
便下襄陽向洛陽。\\
}

\shici{古詩}{風}{唐}{李嶠}{
解落三秋葉,\\
能開二月花。\\
過江千尺浪,\\
入竹萬竿斜。\\
}

\shici{古詩}{涼州詞}{唐}{王翰}{
葡萄美酒夜光杯,\\
欲飲琵琶馬上催。\\
醉臥沙場君莫笑,\\
古來征戰幾人回。\\
}

\shici{古詩}{芙蓉樓送辛漸}{唐}{王昌齡}{
寒雨連江夜入吳,\\
平明送客楚山孤。\\
洛陽親友如相問,\\
一片冰心在玉壺。\\
}

\shici{古詩}{浪淘沙}{唐}{劉禹錫}{
九曲黄河萬里沙,\\
浪淘風簸自天涯。\\
如今直上銀河去,\\
同到牽牛織女家。\\
}

\shici{古詩}{蜂}{唐}{羅隱}{
不論平地與山尖,\\
無限風光盡被占。\\
采得百花成蜜後,\\
爲誰辛苦爲誰甜?\\
}

\shici{古詩}{書湖陰先生壁}{宋}{王安石}{
茅檐長掃淨無苔,\\
花木成畦手自栽。\\
一水護田將綠繞,\\
兩山排闥送青來。\\
}

\shici{古詩}{六月二十七日望湖樓醉書}{宋}{蘇軾}{
黑雲翻墨未遮山,\\
白雨跳珠亂入船。\\
卷地風來忽吹散,\\
望湖樓下水如天。\\
}

\shici{古詩}{秋夜將曉出籬門迎涼有感}{宋}{陸游}{
三萬里河東入海,\\
五千仞嶽上摩天。\\
遺民淚盡胡塵里,\\
南望王師又一年。\\
}

\shici{古詩}{已亥雜詩}{清}{龔自珍}{
九州生氣恃風雷,\\
萬馬齊喑究可哀。\\
我勸天公重抖擻,\\
不拘一格降人才。\\
}

\shici{古詩}{從軍行}{唐}{王昌齡}{
青海長雲暗雪山,\\
孤城遙望玉門關。\\
黄沙百戰穿金甲,\\
不破樓蘭終不還。\\
}

\shici{古詩}{竹枝詞}{唐}{劉禹錫}{
楊柳青青江水平,\\
聞郎江上唱歌聲。\\
東邊日出西邊雨,\\
道是無晴還有晴。\\
}




\shici{古詩}{竹裏館}{唐}{王維}{
獨坐幽篁裏,彈琴複長嘯。\\
深林人不知,明月來相照。
}

\shici{古詩}{送別}{唐}{王維}{
山中相送罷,日暮掩柴扉。\\
春草明年綠,王孫歸不歸?
}

\shici{古詩}{相思}{唐}{王維}{
紅豆生南國,春來發幾枝?\\
願君多采擷,此物最相思。
}

\shici{古詩}{雜詩}{唐}{王維}{
君自故鄉來,應知故鄉事。\\
來日綺窗前,寒梅著花未?
}

\shici{古詩}{八陣圖}{唐}{杜甫}{
功蓋三分國,名成八陣圖。\\
江流石不轉,遺恨失吞吳。
}

\shici{古詩}{送靈澈上人}{唐}{劉長卿}{
蒼蒼竹林寺,杳杳鍾聲晚。\\
荷笠帶斜陽,青山獨歸遠。
}

\shici{古詩}{彈琴}{唐}{劉長卿}{
泠泠七弦上,靜聽松風寒。\\
古調雖自愛,今人多不彈。
}

\shici{古詩}{秋夜寄邱員外}{唐}{韋應物}{
懷君屬秋夜,散步詠涼天。\\
空山松子落,幽人應未眠。
}

\shici{古詩}{問劉十九}{唐}{白居易}{
綠蟻新醅酒,紅泥小火爐。\\
晚來天欲雪,能飲一杯無。
}

\shici{古詩}{渡漢江}{唐}{宋之問(李頻)}{
嶺外音書斷,經冬複曆春。\\
近鄉情更怯,不敢問來人。
}

\shici{古詩}{規雁}{唐}{杜甫}{
東來萬裏客,亂定幾年歸。\\
腸斷江城雁,高高正北飛。
}

\shici{古詩}{絕句}{唐}{杜甫}{
江動月移石,溪虛雲傍花。\\
烏棲知故道,帆過宿誰家。
}

\shici{古詩}{絕句}{唐}{杜甫}{
江碧鳥逾白,山青花欲燃。\\
今春看又過,何日是歸年。
}

\shici{古詩}{鳥鳴澗}{唐}{王維}{
人閑桂花落,夜靜春山空。\\
月出驚山鳥,時鳴春澗中。
}

\shici{古詩}{秋浦歌}{唐}{李白}{
白發三千丈,緣愁似個長。\\
不知明鏡裏,何處得秋霜。
}

\shici{古詩}{尋菊花潭主人不遇}{唐}{孟浩然}{
行至菊花潭,村西日已斜。\\
主人登高去,雞犬空在家。
}

\shici{古詩}{送郭司倉}{唐}{王昌齡}{
映門淮水綠,留騎主人心。\\
明月隨良掾,春潮夜夜深。
}

\shici{古詩}{答武陵太守}{唐}{王昌齡}{
仗劍行千裏,微軀敢一言。\\
曾為大梁客,不負信陵恩。
}

\shici{古詩}{詠史}{唐}{高適}{
尚有綈袍贈,應憐範叔寒。\\
不知天下士,猶作布衣看。
}

\shici{古詩}{絕句}{晉}{陶淵明}{
盛年不重來,一日再難晨。\\
及時當勉勵,歲月不待人。
}

\shici{古詩}{和郭主簿}{晉}{陶淵明}{
芳菊開林耀,青松冠岩列。\\
懷此貞秀姿,卓為霜下傑。
}

\shici{古詩}{殘菊}{宋}{梅堯臣}{
零落黃金蕊,雖枯不改香。\\
深叢隱孤芳,猶得車清觴。
}

\shici{古詩}{雲安九日}{唐}{杜甫}{
寒花開已盡,菊蕊獨盈枝。\\
舊摘人頻異,輕香酒暫隨。
}

\shici{古詩}{偶作}{清}{袁枚}{
偶尋半開梅,閑倚一竿竹。\\
兒童不知春,問草何故綠。
}

\shici{古詩}{易水送別}{唐}{駱賓王}{
此地別燕丹,壯士發沖冠。\\
昔時人已沒,今日水猶寒。
}

\shici{古詩}{秋風引}{唐}{劉禹錫}{
何處秋風至,蕭蕭送雁群。\\
朝來入庭樹,孤客最先聞。
}

\shici{古詩}{梅花}{宋}{王安石}{
牆角數枝梅,淩寒獨自開。\\
遙知不是雪,為有暗香來。
}

\shici{古詩}{梅花}{元}{王冕}{
清苦良自持,忘言養高潔。\\
靜夜月華明,開門滿山雪。
}

\shici{古詩}{送元二使安西}{唐}{王維}{
渭城朝雨浥輕塵,
客舍青青柳色新。
勸君更盡一杯酒,
西出陽關無故人。
}

\shici{古詩}{江畔獨步尋花}{唐}{杜甫}{
黃四娘家花滿蹊,
千朵萬朵壓枝低。
留連戲喋時時舞,
自在嬌鶯恰恰啼。
}

\shici{古詩}{鄉村四月}{宋}{翁卷}{
綠滿山原白滿川,子規聲裏雨如煙。
鄉村四月閑人少,才了蠶桑又插田。
}

\shici{古詩}{墨梅}{元}{王冕}{
吾家洗硯池邊樹,
朵朵花開淡墨痕。
不要人誇好顏色,
只留清氣滿乾坤。
}


\end{document}
